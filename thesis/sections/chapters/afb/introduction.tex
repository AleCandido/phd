% !TeX root = ../../../main.tex

An important direction for ongoing and future studies of  new physics
\acrfull{bsm} at the \acrfull{lhc} is the search for novel heavy resonances.
%
The \lhc is uniquely suited to direct searches for these resonances, thanks to
its unparalleled center of mass energy,
$\sqrt{s}=\SI{13.6}{\tera\electronvolt}$ in the recently started  Run III, and
the high statistics to be accumulated in the coming years, especially in the
\acrfull{hl} phase.
%
For instance, considering representative benchmark \bsm scenarios (cf.\
\cite{CidVidal:2018eel}), the \hl-\lhc is sensitive to  searches for sequential
Standard Model (SM)  $W'$ gauge bosons  up to
$m_{W'}=\SI{7.8}{\tera\electronvolt}$, $E_6$ model $Z'$ gauge bosons up to
$m_{Z'}=\SI{5.7}{\tera\electronvolt}$, and Kaluza-Klein resonances decaying
into a $t\bar{t}$ pair up to $m_{KK}=\SI{6.6}{\tera\electronvolt}$.

The production of such high-mass states proceeds via partonic scattering that
involves large  values of the momentum fractions $x_1$ and $x_2$ of the
colliding partons, because the center of mass energy of the partonic collision
is $\hat s= x_1 x_2 s$.
%
For instance, the on-shell production of a state with invariant mass
$m_{X}=\SI{8}{\tera\electronvolt}$ requires $x_1x_2 \gsim 0.3$, hence for
central production at leading order $x_1=x_2\approx 0.6$. 
This is problematic because \pdfs are poorly known for $x\gsim 0.4$ (cf.
\cite{Gao:2017yyd,Kovarik:2019xvh}), as there is limited data included in
current PDF determinations to constrain this kinematic region.
%
Indeed, in the past, claims of possible \bsm signals~\cite{CDF:1996yow} 
were subsequently traced to poor modeling of the PDFs in the large-$x$
region~\cite{Lai:1996mg}.
%
The impact of lack of knowledge of the \pdfs
on \bsm searches is thus a delicate issue~\cite{Beenakker:2015rna}.

Here we wish to further investigate this by specifically considering
\acrfull{nc} \acrfull{dy} dilepton production and associated observables,
frequently  used for \bsm searches at the \lhc.
%
\nc \acrlong{dy} production is one of the cleanest processes in the search for 
both narrow and broad heavy resonances decaying into dileptons, $pp \to X \to
\ell^+\ell^-$, since  the two charged leptons can be detected with excellent
energy and angular resolution.
%
This also enables the search for smooth, non-resonant distortions with respect
to the SM backgrounds, such as those arising in the context of contact
interactions  or, more generally, induced by \acrfull{eft} higher-dimensional
operators that lead to  direct couplings between quarks and
leptons \cite{Ethier:2021bye,Dawson:2018dxp,Ellis:2020unq,Greljo:2021kvv}.
Indeed, both \atlas and \cms have extensively explored this channel in their
\bsm search program
\cite{ATLAS:2014gys,ATLAS:2020yat,ATLAS:2019erb,CMS:2021ctt,ATLAS:2021mla,CMS:2018nlk}.
To this purpose, it is mandatory to have a detailed understanding of the
dominant \sm background, namely dilepton production from quark-antiquark
annihilation mediated by a virtual \acrfull{ew} boson, $q\bar{q} \to 
\gamma^*/Z \to \ell^+\ell^-$, with sub-leading processes involving the
quark-gluon and photon-photon initial states.

\acrlong{dy} production is one of the SM processes which is known to highest
perturbative accuracy: indeed, both N$^3$LO QCD results~\cite{Duhr:2021vwj} and
the full mixed \qcd-\ew corrections at \nnlo
\cite{Buccioni:2020cfi,Buccioni:2022kgy,Bonciani:2020tvf,Bonciani:2021zzf,Armadillo:2022bgm}
have become available recently.
%
Therefore, the main uncertainty on theoretical predictions for this process is
mostly due to the \pdfs, which, as mentioned, are poorly known at large $x$.
%
Experimentally, uncertainties are minimized when considering observables in
which several systematics cancel in part or entirely.
%
An example relevant for the \dy process is the forward-backward asymmetry
$A_{\text{fb}}$ of the angular distribution of the dilepton pair in the
center-of-mass frame of the partonic collision, i.e.\ the asymmetry in the
so-called Collins-Soper angle $\theta^*$, recently measured from the Run II
dataset by \atlas, \cite{ATLAS:2017rue}, and \cms, \cite{CMS:2022uul}.
%
The sensitivity of this observable to both \pdfs and \bsm signals has
been emphasized recently
\cite{Fiaschi:2021sin,Fiaschi:2021okg,Accomando:2019vqt,Accomando:2018nig}, as
well as its relevance to extractions of the weak mixing angle $\sin^2\theta_W$
at the \lhc~\cite{CMS:2018ktx}.
These studies  are mostly restricted to the vicinity of the $Z$-boson peak,
$m_{\ell\bar{\ell}} \sim m_Z$ with $m_{\ell\bar{\ell}}$ being the dilepton
mass, though in a recent study by CMS~\cite{CMS:2022uul} the forward-backward
asymmetry has been used to obtain a lower mass limit (of
$\SI{4.4}{\tera\electronvolt}$) on a hypothetical $Z'$ heavy gauge boson.

In this work, we assess to which extent different assumptions on the large-$x$
behavior of \pdfs, as well as different estimates of the \pdf uncertainty in
this region, may affect \bsm searches, by specifically studying \nc
\acrlong{dy} production, and the forward-backward asymmetry in particular.
To this purpose, we explain the dependence of the general qualitative features
of the asymmetry on the behavior of PDFs, based on an understanding  of the
analytic dependence of the asymmetry on the partonic luminosities.
%
We then present detailed computations of the forward-backward asymmetry at the
\lhc, with realistic experimental cuts, using a variety of PDF sets.

We find that first, the large-$x$ \pdf shape and uncertainty can differ
considerably between different \pdf sets, with \nnpdfr{4.0},
\cite{Ball:2021leu}, generally displaying a more flexible shape and a wider
uncertainty.
%
And second, that all \pdf sets except \nnpdfr{4.0} lead to a qualitative
behavior of the asymmetry which in the large-mass multi-TeV region reproduces
the shape found around the $Z$-peak region, even though there is no fundamental
reason why this should be the case.
%
We will then trace the observed behavior of the asymmetry to that of the
underlying \pdfs.
