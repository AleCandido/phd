% !TeX root = ../../../main.tex

In this work we have scrutinised the \pdf dependence of  neutral current
Drell-Yan production
at large dilepton invariant masses $\mll$, focusing on the behavior
of the forward-backward asymmetry $A_{\text{fb}}$
in the Collins-Soper angle $\cos\theta^*$, an observable frequently
considered in the context of searches for new physics beyond the SM.
%
We have demonstrated that while theoretical
predictions for the sign and magnitude of $A_{\text{fb}}$ are very
similar for all \pdf sets in the
$Z$ peak region, they
depend markedly on the choice of \pdf set for  large values of $\mll$. 
We have traced this behavior to that of the \pdfs, which agree in the
data region, but differ in the large-$x$
region, where \pdfs are mostly unconstrained by data.

We have specifically shown that the uncertainty on the asymmetry
differs substantially between \pdf sets, with \nnpdfr{4.0} displaying
a more marked increase as  $\mll$ grows, leading to
an absolute uncertainty that e.g.\ for $\mll^{\text{min}}\gtrsim\SI{4}{\tera\electronvolt}$
is about twice as large as that found using CT18, 
four times as large as MSHT20,
and about one order of magnitude larger than ABMP16.
%
Also, whereas other \pdf sets predict a shape of the asymmetry
which is unchanged when  $\mll$ increases from the $Z$-peak region to
the TeV range, namely a 
positive
asymmetry implying a larger cross-section  for $\cos\theta^*\ge 0$, \nnpdfr{4.0} finds that as
$\mll$ increases, the asymmetry is reduced, and the $\cos\theta^*$ distribution
becomes symmetric when $\mll^{\text{min}}\sim\SI{5}{\tera\electronvolt}$.

We have traced this behavior to that of the underlying
\pdfs in the large-$x$ region, where \pdfs are mostly unconstrained by
data.
%
Specifically we have seen that in this region \nnpdfr{4.0} has
generally wider uncertainties.
%
Also, while for  all \pdf  sets the
quark and antiquark distributions vanish as
a power of $(1-x)$ as $x\to 1$, for all groups but \nnpdfr{4.0} this power is
constant for light quarks to the right of the valence peak, while for
\nnpdfr{4.0} it changes as $x$ increases, slowly for up quarks, more rapidly
for down quarks and even more rapidly for antiquarks.
%
All this suggests that the different behavior of
\nnpdfr{4.0} is due to its more flexible \pdf parametrization.

Our general conclusion is that the 
behavior of the forward-backward asymmetry
observed at lower invariant masses is not necessarily reproduced at
large masses if flexible enough
\pdfs are used:  the characteristic positive asymmetry observed
for low $\mll$ values
can be washed out in the high-mass region.
%
Hence, deviations from the traditional expectation of a positive forward-backward
asymmetry in high-mass Drell-Yan cannot be taken as an indication of 
\bsm physics,
at least based on  our current understanding of proton structure in the large-$x$ region.

Turning the argument around, future measurements of the $\cos\theta^*$
distribution and the associated forward-backward asymmetry 
$A_{\text{fb}}$ when included in \pdf determinations could help in
constraining \pdfs at large $x$.
%
For instance, \cref{fig:afb/CMS_DY_14TEV_MLL_others} indicates that for
$\mll^{\text{min}}=\SI{5}{\tera\electronvolt}$ and $\sqrt{s}=\SI{14}{\tera\electronvolt}$ the
asymmetry $A_{\text{fb}}$ can be as large as 50\% for ABMP16
while it vanishes (within large uncertainties) in the case of \nnpdfr{4.0}.
%
By rebinning the $\cos\theta^*$ distribution, for an integrated
luminosity of $\mathcal{L}=6$ ab$^{-1}$, corresponding to the
combination at ATLAS and CMS 
at the end of the HL-\lhc data-taking period, $\mathcal{O}(10)$ events are expected in the backward region,
with an statistical uncertainty of $\delta_{\text{stat}}\sim 30\%$ which could be sufficient to
discriminate between these two limiting scenarios at the $2\sigma$ level.

Higher event counts are expected if the $\mll$ cut is loosened, though one is
then less sensitive to the large-$x$ region where differences between \pdf sets and their
uncertainties are the largest.
%
Ultimately, the constraining power of high-mass Drell-Yan in general and of the forward-backward
asymmetry in particular can only be addressed by means of a dedicated projections
based on binned pseudo-data such as those carried
out for the HL-\lhc and the Electron Ion Collider in e.g.~\cite{AbdulKhalek:2018rok,Khalek:2021ulf}.
%
While we leave this exercise for a future study, the investigations
presented in this work indicate that $A_{\text{fb}}$
at high-invariant masses represents a promising and mostly
unexplored channel to pin down large-$x$ light
quark and antiquark \pdfs at the HL-\lhc.

While in this work
we have focused on the forward-backward asymmetry in neutral-current Drell-Yan production,
similar considerations apply for other processes relevant
for \bsm searches at high mass at the \lhc.
%
Indeed, the HL-\lhc will be sensitive to a broad range of hypothetical
new massive particles, from resonances in the $m_{jj}$ dijet invariant mass distribution up to 11 TeV,
heavy vector triplet resonances decaying into a diboson $VV'$ pair up to 5 TeV,
and gluinos with masses up to $m_{\tilde{g}}=\SI{3}{\tera\electronvolt}$ in the minimal
supersymmetric standard model (MSSM) with a massless lightest SUSY
particle~\cite{CidVidal:2018eel}.
%
For all these channels, a robust understanding of \pdfs
and their uncertainties at large $x$, including the role of
methodological and model assumptions, will be necessary to fully exploit
the HL-\lhc discovery potential for \bsm signatures.
%
%
Conversely, once \bsm phenomena have been excluded in some high-energy channel,
the corresponding search can be unfolded into a measurement to provide direct
constraints on the \pdfs in this key large-$x$ region, which in turn will
enhance the reach of other searches.

\subsection*{Acknowledgments}

We are grateful to Dimitri Bourilkov, Alexander Grohsjean, Meng Lu, and Jan Schulte for raising with us the issue of the
\pdf dependence of $A_{\text{fb}}$ at high invariant masses and for the subsequent discussions.
%
A.~C., S.~F., and F.~H.\ are supported by
the European Research Council under 
the European Union's Horizon 2020 research and innovation Programme
(grant agreement n.740006).
%
R.~D.~B.\ is supported by the U.K.\
Science and Technology Facility Council (STFC) grant ST/P000630/1.
%
E.~R.~N.\ is supported by the Italian Ministry of University and Research (MUR)
through the ``Rita Levi-Montalcini'' Program.
%
J.~R.\ is partially supported by NWO (Dutch Research Council).
%
C.~S.\ is supported by the German Research Foundation (DFG) under
reference number DE~623/6-2.
