% !TeX root = ../main.tex

%************************************************
\chapter{Asymptotic Safety}
\label{cap:AsSty}
\label{cap:assty}
%************************************************
\minitoc
\adjustmtc

%************************************************
\section{The problem of Quantum Gravity}
\label{sec:problemQG}
%************************************************

\newacronym{qm}{QM}{Quantum Mechanics}
\newacronym{gr}{GR}{General Relativity}

An open problem in Theoretical Physics is how to reconcile \acrfull{qm} and \acrfull{gr}.
%\marginpar{\acrshort{qm} and \acrshort{gr}}

The two theories achieved fundamental results in their respective domains\footnote{the domain in which they show relevant differences from the classical theory} (atomic and particle physics for \acrshort{qm}, astrophysics and cosmology for \acrshort{gr}), leading to the confirmation of their validity and their capability in the description of Nature.
\newline

\newglossaryentry{lpl}
{
	type=symbols,
	name={Planck length},
	description={The length scale related to the fundamental constants $ c $, $ \hbar $ and $ G $},
	symbol={\ensuremath{\ell_{Pl}}}
}
\newglossaryentry{mpl}
{
	type=symbols,
	name={Planck mass},
	description={The mass scale related to the fundamental constants $ c $, $ \hbar $ and $ G $},
	symbol={\ensuremath{M_{Pl}}}
}

Moreover it is worth noticing that gravity is quite different from others interaction: it's typical length is the Planck scale ($ \gsym{lpl} = \gsym{mpl}^{-1} = \unit{1.6 \times 10^{-33}}{\centi\meter} $), but it is known from its effects on much larger scales, because at the typical lengths of others interactions it is hidden by them, for its weakness.


\begin{equation}
	\sqrt{2\arcsin 1}=\sqrt[4]{6\sum_{k\ge1}\frac{1}{k^{2}}}=
	\int_{-\infty}^{\infty}e^{-x^{2}}\,dx \quad\text{text}
\end{equation}


%************************************************
\section{Renormalizable theories of gravitation}
\label{sec:ReThGrav}
%************************************************

In \cref{sec:AsSty} it will be described a possible scenario for the solution to the problem of non-renormalizability of the \textit{quantized} theory of gravity, introduced in \cite{Weinberg1979} in the context of QFT.
\newline

Instead in this section some preexisting alternatives are reviewed (highlighted also in \cite{Weinberg1979}), together with some contras for these theories.


\subsection{Extended theories of gravitation}


Considering that the main idea of \acrshort{gr} is to build a theory invariant under diffeomorphisms the request for a successful theory of gravity is to fulfill this requirement.


%************************************************
\section{Asymptotic Safety}
\label{sec:AsSty}
%************************************************

\lipsum


%************************************************
\section{Another Section}
\label{sec:AnSec}
%************************************************

\lipsum

%************************************************
\section{Yet Another Section}
\label{sec:YAnSec}
%************************************************

\lipsum
