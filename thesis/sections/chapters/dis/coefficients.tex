% !TeX root = ../../../main.tex

Using the collinear factorization theorem of \dis, \cite{Collins:1989gx}, we
can write any hadronic structure function $F_k$ in terms of \pdf
$f_j(\xi,\mu_F^2)$ and the coefficient functions $c_{j,k}(z,
Q^2,\mu_F^2,\mu_R^2)$ (acting as \textit{partonic structure functions}) using a
convolution over the first argument:
\begin{equation}
    F_k^{bb'}(x,Q^2,\mu_F^2,\mu_R^2) = \sum_{p} f_p(\mu_F^2) \otimes_x c_{k,p}^{bb'}(Q^2,\mu_F^2,\mu_R^2)
\end{equation}
where the sum runs over all contributing partons $p\in\{g,q,\bar q\}$. In the
following we will assume that a quark $\hat q$ is hit by the boson. Note that
this is \textit{independent} of the incoming parton $p$.
%
The dependency on the renormalization and factorization scales has to be
propagated consistently, in order to be able to use them as an estimate for
\mhou (cf. \cref{ch:mhou}).
For those cases in which scale variations are not relevant, it is safe to
consider $\mu_R^2 = \mu_F^2 = Q^2$.

Using \pqcd, it is possible to expand the coefficient functions in powers of
the strong coupling $a_s(\mu_R^2) = \frac{\alpha_s(\mu_R^2)}{4\pi}$:
\begin{equation}
    c_{k,p}^{bb'}(z, Q^2,\mu_F^2,\mu_R^2) = \sum_{l=0} a_s^l(\mu_R^2) c_{k,p}^{bb',(l)}(z, Q^2,\mu_F^2,\mu_R^2)
\end{equation}
In practice, different normalization might be used.

Similar to the splitting on the leptonic side we have to split on the partonic
side again:
\begin{align}
    c_{k,p}^{bb'} &= g_{\hat q,b}^V g_{\hat q,b'}^V c_{k,p}^{VV} + g_{\hat q,b}^A g_{\hat q,b'}^A c_{k,p}^{AA} \qquad~ k\in\{1,2,L\} \\
    c_{3,p}^{bb'} &= g_{\hat q,b}^V g_{\hat q,b'}^A c_{3,p}^{VA}
\end{align}

The main categories for coefficients the same of Structure Functions, i.e.:
\begin{itemize}
\item the \textbf{process} (\em/\nc/\cc)
\item the \textbf{kind} (F2/L/3)
\item the \textbf{heavyness} involved; it slightly differ from that of
  structure functions, since it is referred to individual contributions
  \begin{itemize}
    \item \textit{light}, when no mass is involved
    \item \textit{heavy}, when mass effects are accounted for, with a single
      quark mass (two mass effects are rather negligible, and very complex to
      include), since (it is the same for every flavor, just depending on the
      numerical value of the mass as a parameter)
    \item \textit{asymptotic}, that are the limit of heavy contributions, to
      subtract the double counting in \gmvfns schemes, like \fonll
    \item \textit{intrinsic}, in which the incoming parton is a massive one
      (can also combine with asymptotic)
  \end{itemize}
\item but there is a new one: the \textbf{channel} (ns/ps/g), and it is related to the
  incoming parton:
  \begin{itemize}
    \item if the \ew boson it is coupling to a \textit{quark} line connected to the incoming
      one, than each PDF it's contributing proportionally to his charge (e.g.:
      electric charge for the photon); this is called \textbf{non-singlet (ns)}
    \item otherwise the line to which the \ew boson is coupling it will be detached
      from the incoming  by \textit{gluonic} lines, and the gluon is flavor blind, so
      all the charges are summed and all the PDF are contributing the same way;
      this is called \textbf{pure singlet (ps)}
    \item eventually: if a \textit{gluon} is entering all the quarks will couple to the \ew
      boson (if no further restrictions are imposed by the observable, e.g.
      F2charm), as in the singlet case, and so the charges are summed over; this
      is called the \textbf{gluon (g)} (because \textit{it is} the gluon...)
    \item the \textbf{parity structure}
      (vectorial-vectorial/axial-axial/vectorial-axial), it is relevant only for
      the NC, and should be taken into account
  \end{itemize}
\end{itemize}
Of course, there the coefficient functions also depends on the
\textbf{perturbative order} they are computed at.

A recap of the status of coefficient functions as implemented in \yadism (cf.
\cref{sec:dis/yadism}), is contained in \cref{tab:dis/coefffuncs}.


%%%%%%%%%%%%%%%%%%%%%%%%%%%
\renewcommand{\thefootnote}{\fnsymbol{footnote}}
\newcounter{numfootnote}
\setcounter{numfootnote}{\value{footnote}}
\setcounter{footnote}{0}

%%%%%%%%%%%%%%%%%%%%%%%%%%%%%%%%%%%%%%%%%%%%%%%%%%%%%
\begin{table}
  \label{tab:coefffuncs}
  \centering
   \renewcommand{\arraystretch}{1.30}
   \renewcommand{\tabularxcolumn}[1]{w{c}{#1}}
   \begin{tabularx}{\textwidth}{X | c c c c}
      \toprule
      $\quad$ NLO$\quad$ & $\quad$light$\quad$ & $\quad$heavy$\quad$ & $\quad$intrinsic$\quad$ &$\quad$ asymptotic $\quad$ \\
      \hline
      NC & \grokcell & \grokcell & \grokcell & \grokcell\\
      CC & \grokcell & \grokcell & \grokcell & \grokcell\\
      \midrule
      NNLO & & &\\
      \hline
      NC & \grokcell & \grokcell & \rdxcell & \grokcell\\
      CC & \grokcell & \ylcell tabulated\footmark{1} & \rdxcell & \grokcell\\
      \midrule
      N3LO & & &\\
      \hline
      NC & \grokcell &  \rdxcell\footmark{2} & \rdxcell & \rdxcell\footmark{3} \\
      CC & \grokcell &  \rdxcell\footmark{2} & \rdxcell & \rdxcell \\
      \bottomrule
   \end{tabularx}
  {
    \footnotesize
    \begin{tabularx}{\textwidth}{w{r}{2em} p{40em}}
      \fnsym{1} & Already available as $K$-factors~\cite{Gao:2017kkx}, now being integrated in the grid format.\\
      \fnsym{2} & Full calculation not available but an approximated expression can
      be constructed from partial results~\cite{niccolo}.\\
      \fnsym{3} & Calculation available, to be implemented.
    \end{tabularx}
  }
  \vspace{0.2cm}
  \caption{Overview of the different types and accuracy of the DIS coefficient
    functions currently implemented in \yadism. For each perturbative order (NLO, NLO, and N3LO)
    we indicate  the light-to-light (``light''), light-to-heavy (``heavy''), heavy-to-light
    and heavy-to-heavy (``intrinsic'') and ``asymptotic'' ($Q^2 \gg m_h^2$ limit)  coefficients functions
    which have been implemented and benchmarked.
    %
    The NNLO heavy quark coefficient functions for CC scattering are available in $K$-factor format
    and are being implemented into the \yadism grid formalism.  \label{tab:coefffuncs}
  }
\end{table}
%%%%%%%%%%%%%%%%%%%%%%%%%%%%%%%%%%%%%%%%%%%%%%%%%%%%%%%%%%%%%%%%%%%%%%%%%%%%%%%%%%%%%%%%%%

\renewcommand*{\thefootnote}{\arabic{footnote}}
\setcounter{footnote}{\value{numfootnote}}

%%%%%%%%%%%%%%%%%%%%%%%%%%%

\subsection{Treatment of distributions}
\label{sec:dis/distr}

Coefficient functions are not always pure functions of the partonic variables,
since in absence of masses, regulating all divergences, the regularization
itself can generate distributions.
%
All distributions disappears once convoluted with the \pdf (or a suitable
interpolation basis, as a placeholder for a generic \pdf):
\begin{equation}
    \sigma = \sum_j f_j \otimes c_j = \sum_j \int\limits_x^1 \frac {dz}{z} f_j(x/z) c_j(z)
\end{equation}
so they never survive in physical observables.

A generic coefficient function will allow for three ingredients:
\begin{itemize}
  \item \textit{Regular} functions $r(z)$ that are well behaving, i.e. integrable,
    for all $z \in (0,1]$; these typically contain polynomials, logarithms
    and dilogarithms
  \item \textit{Dirac-$\delta$} distributions: $\delta(1-z)$
  \item \textit{Plus} distributions: $\left[g(z)\right]_+$ which have a
    regulated singularity at $z\to 1$ and are defined by
\end{itemize}

\begin{equation}
  \int\limits_0^1 \!dz\, f(z) \left[g(z)\right]_+ = \int\limits_0^1\!dz\, \left(f(z) - f(1)\right)g(z)
\end{equation}

The \enquote{plused} function can be a generic function, but in practice will
almost always be $\log^k(1-z)/(1-z)$.
The \enquote{plused} function has to be regular at $z=0$.
These contributions are related to soft and/or collinear singularities in the
physical process.

In order to do the convolution in a generic way we adopt the \acrfull{rsl}
scheme: i.e. we categorize them by their behavior under the convolution
internal.
%
This is needed because of the mismatch in the definitions of the convolution
and the plus prescription. Any coefficient function $c(z)$ can be written
in the following way:
\begin{equation}
  f \otimes c = \int\limits_x^1 \! \frac{dz}{z} \, f(x/z) c^R(z) + \int\limits_x^1 \! dz \, \left(\frac{f(x/z)}{z} - f(x)\right) c^S(z) + f(x) c^L(x)
\end{equation}

The remapping of the coefficient function ingredients on to the \rsl elements
is done in the following way:
\begin{itemize}
  \item Regular functions $c(z) = r(z)$ contribute only to the regular bit:
    \begin{equation}
      c^R(z) = r(z)\,,\quad c^S(z) = 0 = c^L(x)
    \end{equation}

  \item Dirac delta distributions $c(z) = \delta(1-z)$ only contribute to
    the local bit:
    \begin{equation}
      c^R(z) = 0 = c^S(z)\,,\quad c^L(x) = 1
    \end{equation}

  \item "Raw" plus distributions $c(z) = \left[g(z)\right]_+$ contribute to
    both the singular and the local bit:
    \begin{equation}
      c^R(z) = 0\,,\quad c^S(z) = g(z)\,,\quad c^L(x) = -\int\limits_0^x\!dz\, g(z)
    \end{equation}
    derivation
    \begin{align}
        f \otimes [g]_+ &= \int\limits_x^1 \frac{dz}{z} f(x/z) \cdot \left[ g(z) \right]_+\\
          &= \int\limits_0^1 \frac{dz}{z} f(x/z) \cdot \left[ g(z) \right]_+ - \int\limits_0^x \frac{dz}{z} f(x/z) \cdot \left[ g(z) \right]_+\\
          &= \int\limits_0^1\!dz\, \left(\frac{f(x/z)}{z} - f(x)\right) \cdot g(z) - \int\limits_0^x\!dz\, \frac{f(x/z)}{z} \cdot g(z)\\
          &= \int\limits_x^1\!dz\, \left(\frac{f(x/z)}{z} - f(x)\right) \cdot g(z) - f(x) \int\limits_0^x\!dz\, g(z)\\
          &\qquad\Rightarrow
          \begin{cases}
            c^R(z) &= 0\\
            c^S(z) &= g(z)\\
            c^L(x) &= -\int\limits_0^x\!dz\, g(z)
          \end{cases}
    \end{align}

  \item A product of a regular function and a plus distribution $c(z) = r(z)\left[g(z)\right]_+$
    contributes to all three bits:
    \begin{equation}
      c^R(z) = (r(z)-r(1))g(z)\,,\quad c^S(z) = r(1)g(z)\,,\quad  c^L(x) = -r(1)\int\limits_0^x\!dz\, g(z)
    \end{equation}
    derivation
    \begingroup
    \allowdisplaybreaks
    \begin{align}
        f\otimes c &= \int\limits_x^1 \frac{dz}{z} f(x/z) r(z) \cdot \left[ g(z) \right]_+\\
          &= \int\limits_0^1 \frac{dz}{z} f(x/z) r(z) \cdot \left[ g(z) \right]_+ - \int\limits_0^x \frac{dz}{z} f(x/z) r(z) \cdot \left[ g(z) \right]_+\\
          &= \int\limits_0^1\! dz \left(\frac{f(x/z)r(z)}{z} - f(x)r(1)\right) \cdot g(z) - \int\limits_0^x\!dz\, \frac{ f(x/z) r(z)}{z} \cdot g(z)\\
          &= \int\limits_x^1\! dz \left(\frac{f(x/z)r(z)}{z} - f(x)r(1)\right) \cdot g(z) - f(x) r(1) \int\limits_0^xdz~ g(z)\\
          &= \int\limits_x^1\! dz \left(\frac{f(x/z)(r(z)+r(1)-r(1))}{z} - f(x)r(1)\right) \cdot g(z) - f(x) r(1) \int\limits_0^xdz~ g(z)\\
          &= \int\limits_x^1\! dz \left(\frac{f(x/z)}{z} - f(x)\right) r(1)\cdot g(z) 
          + \int\limits_x^1\! dz \frac{f(x/z)(r(z)-r(1)))}{z} g(z) \nonumber\\
          &\qquad\qquad\qquad- f(x) r(1) \int\limits_0^x\!dz~ g(z)\\
          &= \int\limits_x^1 \frac{dz}{ z} f(x/z)  r(1)\cdot \left[g(z)\right]_+ + \int\limits_x^1\! dz \frac{f(x/z)(r(z)-r(1)))}{z} g(z)\\
          &\qquad\Rightarrow
          \begin{cases}
            c^S(z) &= r(1)g(z)\\
            c^R(z) &= (r(z)-r(1))g(z)\\
            c^L(x) &= -r(1)\int\limits_0^x\!dz\, g(z)
          \end{cases}
    \end{align}
    \endgroup
\end{itemize}

Also consider that a plus distribution that contains a regular and a singular
function $c(z) = \left[r(z)g(z)\right]_+$ can be simplified by
\begin{equation}
  \left[r(z)g(z)\right]_+ = r(z) \left[g(z)\right]_+ - \delta(1-z) \int\limits_0^1 dy~ r(y) \left[g(y)\right]_+
\end{equation}
derivation;
\begin{align}
    \int\limits_0^1 \!dz~ f(z) \left[r(z)g(z)\right]_+ &= \int\limits_0^1 dz \left(f(z) - f(1)\right)r(z)g(z)\\
      &= \int\limits_0^1 \left(f(z)r(z) - f(1)r(1)\right)g(z)~dz - f(1)\int\limits_0^1\! dz(r(z)-r(1))g(z)\\
      &= \int\limits_0^1\! dz~ f(z)\left(r(z) \left[g(z)\right]_+\right) - f(z)\left(\delta(1-z)\int\limits_0^1\! dy~ r(y) \left[g(y)\right]_+\right)
\end{align}
