% !TeX root = ../../../main.tex

\fns or Heavy Quark Matching Schemes are dealing with the ambiguity of
including massive quark contributions to physical cross sections. 
%
In general, it is possible to consider two different kinematic regimes that
require a different handling of the massive contributions: for $Q^2 \lesssim
m^2$ the heavy quark should be treated with the full mass dependence. 
$Q^2 \gg m^2$ however the quark should be considered massless, because
otherwise a resummation of the occurring terms $\log(m^2/Q^2)$ would be
required.

\subsection[Fixed]{\acrlong{ffns}}
\label{sec:dis/ffns}

As the name \ffns suggests, we are considering a fixed number of flavors
$n_f=n_l+1$ with $n_l$ light flavors and \textbf{one} (and \textit{only one})
heavy flavor\textit{} with a finite mass $m$.
The number of light quarks $n_l$ is arbitrary but fixed and can range between
$3$ and $6$. 
Except for intrinsic contributions we are \textit{not} allowing the heavy \pdf
to contribute (and those corresponding to flavors not in the scheme as well).
%
This scheme is adequate for $Q^2\sim m^2$.

\begin{itemize}
\item the \textbf{light} structure functions corresponds to the interaction of
  the purely light partons, i.e.\ the coefficient functions may only be a
  function of $z,~Q^2$ (and unphysical scales); in particular they can
  \textit{not} depend on any quark mass.
  This may be consistently obtained computing contributions for a Lagrangian
  with all masses set to $0$.
  \begin{itemize}
    \item this definition is consistent with
      \cite{Vermaseren:2005qc,Moch:2004xu,Moch:1999eb,Moch:2007rq,Moch:2008fj},
      and \qcdnum, \cite{Botje:2010ay}
    \item but is not consistent with \apfel, \cite{Bertone:2013vaa}, which
      instead is calling \textit{light} the sum of contributions in which a
      light quark is coupled to the \ew boson
  \end{itemize}

\item as noted in \cref{sec:dis/defs}, the \textbf{total} structure functions
  are \textit{not} the sum of \textbf{light} and the single \textbf{heavy}
  ones, but contains additional terms \textbf{Fmissing} such as the Compton
  diagrams in \cite{Hekhorn:2019nlf}; this is the proper physical object,
  accounting for all contributions coming from the full Lagrangian.

\item the \textbf{heavy} structure functions are defined by having in the
  Lagrangian \textit{only} the \ew charges that are associated to the specific
  quark active (the only massive one). In \nc this corresponds to the electric
  and weak charges of the quarks but in \cc the situation is bit more involved:
  we divide the \ckm-matrix into several parts:
  \begin{equation}
      V_{CKM} =
      \begin{pmatrix}
          {\color{Bittersweet}V_{ud}} & {\color{Bittersweet}V_{us}} & {\color{ForestGreen}V_{ub}}\\
          {\color{blue}V_{cd}} & {\color{blue}V_{cs}} & {\color{ForestGreen}V_{cb}}\\
          {\color{purple}V_{td}} & {\color{purple}V_{ts}} & {\color{purple}V_{tb}}
      \end{pmatrix}
  \end{equation}
  and associate the {\color{blue}blue} couplings to the charm structure
  functions, {\color{ForestGreen}green} to bottom and {\color{purple}purple} to top.
  For $F^{\nu,p}_{2,{\color{blue} c}}$ this in effect amounts to
  \begin{align}
    F^{\nu,p}_{2,{\color{blue} c}} &=
      2x\Big\{C_{2,q}\otimes\Big[{\color{blue}V_{cd}}^2(d+\overline{c}) +
        {\color{blue}V_{cs}}^2 (s+\overline{c})\Big]\nonumber\\
      &\qquad\qquad\qquad\qquad\qquad +
        2\left({\color{blue}V_{cd}}^2+{\color{blue}V_{cs}}^2\right)C_{2,g}\otimes g\Big\}
  \end{align}
  Note that even heavier contributions are \textit{not} available.
  E.g.:
  \begin{itemize}
    \item there is no contributions coming from either \textit{bottom} or
      \textit{top} to $F_{2,{\color{blue} c}}$
    \item while \textit{charm} would contribute to $F_{2,{\color{ForestGreen}
      b}}$, but only as a massless flavor.
  \end{itemize}
\end{itemize}


\subsection[Zero-Mass]{\acrlong{zmvfns}}
\label{sec:dis/zmvfns}

As the name \zmvfns suggests, this scheme takes into account a variable number
of light flavors $n_f$ with $n_f = n_f(Q^2)$. 
%
There is an \textit{activation} scale $Q_{thr, i}^2$ associated to each
potentially \enquote{heavy} quark (i.e.\ charm, bottom, and top) and whenever
$Q^2 \ge Q_{thr, i}^2$ this quark is considered massless, otherwise infinitely
massive.

This scheme is adequate for $Q^2\gg m^2$.

\begin{itemize}
\item the \textbf{heavy} structure functions are \textit{not} defined, as quark
  masses are either $0$ or $\infty$ (so no massive correction is available at
  all)
\item \textbf{total} ones thus are equal to \textit{light}
\item \textbf{light} structure functions corresponds to the interaction of the
  purely light partons, i.e.\ the coefficient functions may only be a function
  of $z,Q^2$ and eventually unphysical scales; so they can \textit{not} depend
  on any quark mass
\end{itemize}

\zmvfns dependence on thresholds is simple, they just define the $Q^2$
patches in which $n_f$ is constant (and they are of course different from
the quark masses, that are always considered to be zero or infinite).
Also note that $Q_{thr,i}^2$ are not necessarily, but usually chosen to be, the
quarks' masses.

\subsection[FONLL]{\acrfull{fonll}}
\label{sec:dis/fonll}

\fonll \cite{Forte:2010ta} is a \gmvfns that includes parts of the \dglap
equations into the matching conditions.
That is: two different schemes are considered, and they are matched at a given
scale, accounting for the resummation of collinear logarithms, but also for
power suppressed terms from massive corrections at the same time.
%
In the original paper the prescription is only presented for the charm
contributions, but just as a placeholder of an arbitrary massive quark.
%
The prescription defines two separate regimes, below and above the
\textit{next} heavy quark threshold: $Q_{thr,n_f+2}$.
%
As in the case of \zmvfns, these matching thresholds are not necessarily, but
usually chosen to be, the quarks' masses.

\begin{itemize}
  \item for \textbf{$Q^2 < Q_{thr,n_f+2}^2$}: the general expression,
    eqs.~(14-15) of \cite{Forte:2010ta}, is:
    \begin{align}
        F^{\text{FONLL}}(x, Q^2) = F^{(d)}(x, Q^2) + F^{(n_f)}(x, Q^2)\\
        F^{(d)}(x, Q^2) = F^{(n_f + 1)}(x, Q^2) - F^{(n_f, 0)}(x, Q^2)
        \label{eq:dis/fonll}
    \end{align}

    Here the scheme change between the schemes with $n_f$ (i.e.\ the \ffns
    scheme in which the active flavor is the only one considered to be massive)
    and $(n_f + 1)$ flavors (i.e.\ the \ffns scheme with only massless quarks,
    including the formerly active one) is explicitly included.

    This scheme change is related to the \dglap matching conditions: in
    particular the massive corrections are only coming from the $n_f$
    scheme, but the collinear contribution is present in both:
    \begin{itemize}
      \item the $n_f$ scheme includes the logarithms of the active mass,
          while the \pdf of the massive object are scale-independent by definition
          (since the factorization terms are kept in the matrix element)
      \item the $(n_f + 1)$ scheme does not account for them in them in the coefficient
          function, but instead they are resummed in the \pdf evolution through the
          \dglap equation
    \end{itemize}

    By matching the two schemes a \gmvfns is obtained, accounting for both the
    massive corrections and the resummation of collinear logarithms.
    %
    The matching is obtained subtracting the asymptotic massless limit of the
    massive expression, namely $F^{(n_f, 0)}(x, Q^2)$, while adding the
    $(n_f + 1)$ expression, such that for large $Q^2$ the massive
    $n_f$ contribution cancels with the asymptotic one, and only the truly
    light contribution survives.
    %
    Actually below the former threshold, so $Q^2 < Q_{thr,n_f+1}^2$, \fns
    with $n_f$ flavors is employed, i.e.\ a $\theta(Q^2 - Q_{thr,n_f+1}^2)$ is
    prepended to $F^{(d)}$.

  \item \textbf{above} this threshold: the \zmvfns is employed and this leads
    to an inconsistency at this $Q_{thr,n_f+2}$ threshold, but a good
    approximation nevertheless.
    %
    This amounts to simply make an hard cut to the original smooth decay of
    massive contributions, and to add the subsequent thresholds for the following
    massive quarks.
\end{itemize}

\subsubsection{Damping}
\label{sec:dis/fonll-damp}

Up to \nlo the scheme change (from $n_f - 1$ flavors to $n_f$) is continuous,
but in general it is not.
%
In order to recover the continuous transition a damping procedure may be
adopted, turning the scheme in the so called \textbf{damp FONLL}.

Continuity on its own is not an issue, but it is one symptom of a feature of
$F^{(d)}$ \cref{eq:dis/fonll}: while it improves the behavior at large $Q^2$ it
is unreliable for $Q^2 \sim Q_{thr,n_f+1}^2$.
%
For this reason might be a good idea to suppress $F^{(d)}$ near threshold, and
then this restore continuity.
%
The generic shape of this suppression is written in eq.~(17) of
\cite{Forte:2010ta}, and it is:
\begin{equation}
   F^{(d, th)} (x, Q^2) = f_{\textrm{thr}} (x, Q^2) F^{(d)}(x, Q^2)
\end{equation}
In particular the following conditions are needed for $f_{\text{thr}} (x, Q^2)$
to fit the task:
\begin{itemize}
  \item be such that $F^{(d, th)} (x, Q^2)$ and $F^{(d)} (x, Q^2)$ is
    power suppressed for large $Q^2$
  \item enforce the vanishing of $F^{(d, th)} (x, Q^2)$ at and below threshold
\end{itemize}

A common shape for $f_{\textrm{thr}} (x, Q^2)$ is then:
\begin{equation}
   f_{\textrm{thr}} (x, Q^2) = \theta(Q^2 - m^2) \left(1 -  \frac{Q^2}{m^2}\right)^2
\end{equation}
The power used here is $2$, but in general this is arbitrary, and thus it is a
user choice in \yadism.

\subsubsection{Threshold different from heavy quark mass}
\label{sec:dis/thr-neq-mass}

The threshold in \fonll plays a relevant role, since it is deciding where (in
$Q^2$) the match should happen.
%
A typical choice is to put the threshold on top of the relevant quark mass (also
in \zmvfns, mimicking the opening of a new channel). This is \textit{}not
mandatory\textit{}, as the threshold is just an \fns parameter it can be freely
chosen.
%
If the threshold is then chosen \textit{different} from the quark mass, a new scale
ratio appears, and the expressions might depend also on this one.
Notice that the threshold is only a parameter of \fonll, so it can not affect
the \ffns ingredients of the scheme (which can only depend on the real quark
masses, through massive propagators).
Then only the massless limit (the double counting preventing bit) might include
a threshold dependency, and in practice it will only change the relevant
logarithm, that:
\begin{itemize}
  \item  instead of being the logarithm of the ratio between the process scale
    and \textit{the mass}
  \item is a logarithm of the ratio with \textit{the threshold}
\end{itemize}
as it is discussed in \cite{Forte:2018ovl}.
