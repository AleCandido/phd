% !TeX root = ../../../main.tex

The \acrfull{dis} process has been introduced in \cref{sec:qcd/dis}, briefly
outlining its relevance in the \pdf determination, and enumerating the various
experimental source of \dis data for \pdf fits.

In this chapter, the theory of \dis will be described in more details in
\cref{sec:dis/defs}, defining the related kinematics and physical observables,
in a wide variety of variants.
Then, in \cref{sec:dis/coeffs}, it will follow a brief review of the analytic
ingredients that is possible to compute in perturbation theory, the so-called
\textit{coefficients functions}, with a focus on their analytic properties,
since they are crucial to standardize the presentation, in order to allow fully
automated numerical integration.
It follows a summary of \acrlong{fns} in \cref{sec:dis/fns}, since the
treatment of quark mass effects is crucial to predict the outcome of \dis
experiments, traditionally operating at not-so-high energies (for which charm
and bottom mass effects are most relevant).

Finally, \yadism will be presented, a new program to compute \dis grids and
predictions, that implements all the elements discussed in the preceding
sections, and a range of further features, including extremely relevant ones,
like scale variations (that will be described in more details in
\cref{ch:mhou}, but not specifically in the context of \dis), and many other,
e.g. \tmc and alternative quark mass definitions schemes.

This chapter is not a full review of \yadism development, since it is partly
still work in progress, despite the feature parity with predecessors has
already been reached.
%
The online documentation, intended to be a living document, it is already
publicly available at:
\begin{center}
  \url{https://yadism.readthedocs.io/}
\end{center}
and contains a superset of (almost) all the material in this chapter.
%
Together with the other authors, we hope to present the full set of features in
a future publications, \cite{yadism}.
