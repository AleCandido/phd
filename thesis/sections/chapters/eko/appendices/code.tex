An \eko{} is effectively a rank $5$ tensor $E_{\mu,i \alpha j \beta}$, that
evolves a PDF set from one given scale to a user specified list of final scales
$\mu$:
\begin{equation}
    f_{\mu,i\alpha} = E_{\mu,i \alpha j \beta} f^{(0)}_{j \beta}
\end{equation}
where $i$ and $j$ are indices on flavor, and $\alpha$ and $\beta$ are indices on the $x$-grid.

The computation of each rank $4$ tensor is almost independent:
In a \ffns{} for each target $\mu_F^2$ an operator $\vb{\tilde E}(\mu_{F,0}^2 \to \mu_F^2)$ is computed separately.
Instead, in a \vfns{} first a set of operators is computed, to evolve from the
initial scale to any matching scales (we call these \textit{threshold
operators}). Then, for each target $\mu_F^2$, an operator is computed
from the last intermediate matching scale to $\mu_F^2$; finally
they are composed together.

\subsection{Performance Motivations and Operator Specificity}
\label{app:code:motiv}

Before diving into the details of \eko{} performances there is a fundamental
point that has to be taken into account: \eko{} is somehow unique as an
evolution program, because its main and only output consists in evolution
\textit{operators}.

For this reason, a close comparison on performances with other evolution codes
(whose main purpose is the evolution of individual \pdfs) would be rather
unfair: evolving a single \pdf{} is comparable to the generation of the
transformation of a single direction in the \pdf{} space, while the operator acts
on the whole function space.

The motivation to primarily look for the operator itself relies on the specific
needs of a \pdf{} fit itself.
Indeed, a fit requires repeated usage of evolution for the $\chi^2$ evaluation
for each fit candidate, and a final evolution step for the generation of the
\pdf{} grids to deliver, as those used by \lhapdf{}~\cite{Buckley:2014ana}.
The first step has been automated long ago, by the generation of the
\fk{} tables (formerly done with \apfel{} evolution, through
\href{https://github.com/NNPDF/apfelcomb}{\apfelcomb}, inspired to
\cite{Bertone:2016lga}), that store \pdf{} evolution into the grids for
predictions, while the second was repeated at any fit, since for each fit is a
one-time operation (even though it is actually repeated for the number of Monte
Carlo replicas, or Hessian eigenvectors, whose typical sizes are reported in
\cref{tab:pdfmem}).

\begin{table}
    \centering
    \begin{tabular}{lr}
        \pdf{} set name & members \\
        \hline
        \texttt{CT18NNLO}~\cite{Hou:2019efy} & 59\\
        \texttt{MSHT20nnlo\_as118}~\cite{Bailey:2020ooq} & 65\\
        \texttt{NNPDF40\_nnlo\_as\_01180}~\cite{NNPDF:2021njg} & 101
    \end{tabular}
    \caption{Selected \pdf{} sets with their respective number of members}
    \label{tab:pdfmem}
\end{table}

Actually, both the operations of including evolution in theory grids and \pdf{}
grids generation can be further optimized, considering that the evolution only
depends on a small number of theory parameters, and so the operator does, such
that it can be generated only once, and then used over and over.

On top of replicas generation, the search towards an optimal fitting
methodology is an iterative process, that involves a large number number of fits.
Moreover, whatever program supports the generation of \fk{} tables has to
create some kind of evolution operator on its own (since the goal of \fk{}
tables is exactly to be \pdf{} agnostic).

So, the \eko{} work-flow is not a complete novelty, since it was preceded by
\apfel{} in-memory operator generation, but it is a further and strong
improvement in that direction: being operator-oriented from the beginning,
optimizations have been performed for this specific task\footnote{
E.g.\ internally integrating the minimal amount of anomalous dimensions required
for the operator determination, while still providing a flexible delivery on
all the output dimensions (re-interpolating the $x$ dependencies, or rotating
into different flavor bases).
}, and maintaining an
actual operator format, the operators reuse is possible even across the
boundary of \fk{} tables generation, and applied with benefit, e.g., for the
massive replicas set evolution (consider
\texttt{NNPDF40\_nnlo\_as\_01180\_1000}, that is a single set consisting of
1000 replicas, that can be evolved with a single operator instead of running
1000 times an evolution program, like all the other similar sets), but even
repeated fits.

While the benefit is limited for other use cases, any other highly iterative
phenomenological study, in which \pdf{} evolution is repeatedly evaluated from
different border conditions, would benefit from being backed by \eko{}, since
the cost of \dglap{} evaluation is paid only once (even though we are conscious
that this is mainly beneficial for \pdf{} fitting).

\subsection{Computation time}
\label{app:code:time}

As we said above the computation almost happens independently for each target $\mu_F^2$
and the amount of time required scales almost linearly with the number of requested
$\mu_F^2$, except for the thresholds operators in \vfns{} that are computed only once.

We call computing an operator with a fixed number of flavors \enquote{evolving
in a single patch}, since in a \vfns{} the evolution might span multiple patches.
When more than a single patch is involved, operators have to be joined at
matching scales $\mu_h$ with a non-trivial matching, that has to be
computed separately (these are part of the threshold operators).

Typical times required for these calculations in \eko{} are presented in
\cref{tab:time}.
As expected the complexity of the calculation grows with perturbative order,
and so even the time taken increases.
At \lo{} no matching conditions are needed, while for \nlo{} and
\nnlo{} they are computed once for each matching scale. 

\begin{table}[h]
    \centering
    \begin{tabular}{l|cc}
        & patch & matching \\
        \hline
        \lo{} & $\SI{10}{s}$ & $\varnothing$ \\
        \nlo{} & $\SI{45}{s}$ & $\SI{65}{s}$ \\
        \nnlo{} & $\SI{60}{s}$ & $\SI{75}{s}$ \\
    \end{tabular}
    \caption{Rough estimates of times taken by \eko{}, with an average sized
    $x$-grid of $50$ points and single core.}
    \label{tab:time}
\end{table}

We consider these time performances satisfactory, even though it is not
straightforward to compare \eko{} with the other evolution codes, as mentioned
in \cref{app:code:motiv}. As an example, \nnlo{}
evolution in $\mu_F^2 = \SI[parse-numbers=false]{1.65^2}{Gev^2} \to
\SI{100}{GeV^2}$ crossing the bottom matching at
$\SI[parse-numbers=false]{4.92^2}{Gev^2}$ takes $\sim \SI{60}{s} + \SI{135}{s}$
in \eko{} ($\SI{135}{s}$ for the thresholds operators initialization,
$\SI{60}{s}$ for the last patch). \apfel{} takes $\sim\SI{25}{s}$ on the same
custom interpolation $x$-grid (\apfel{} is able to perform significantly better
on a pre-defined, built-in grid).

This comparison shows that on the evolution of a single PDF \eko{} is not
really competitive, but the ratio is limited to $\sim 7.5$. 
However, we already pointed out that the two programs perform a rather
different task: computing a whole operator against a single \pdf{} evolution
(on which the benchmarking is done, only because both programs are able to
perform this simple task, but it is a worthless task for \eko{} usage).

The comparison is technically possible, but we do not encourage this kind of
benchmarks, because the typical task is actually different, and this motivates
the different performances.
\eko{} perform bad in the case of the single task, but with a perfect scaling
(negligible work needed for repeated evolution, practically constant), while
any other program would perform better for the atomic task, but with a linear
scaling in the number of objects to be evolved.

Each program should be selected having in mind the specific usage. \eko{} is
recommended for \pdf{} fitting, and repeated evolution in general.

The time measures in \cref{tab:time} and the rest of this
section have been obtained on a regular consumer laptop:
\begin{verbatim}
OS: Linux 5.13 Ubuntu 21.10 21.10 (Impish Indri)
CPU: (4) x64 Intel(R) Core(TM) i5-6267U CPU @ 2.90GHz
Memory: 7.56 GB
\end{verbatim}
None one of them is a careful benchmark, i.e.\ repeated multiple times, but is mainly
meant to show an order of magnitudes comparison.

\subsection{Memory footprint}
\label{app:code:memory}

Memory usage is dominated by the size of the final object produced,
since a much smaller internal representation is used during the computation.
The final object holds information about the rank $5$ operator, so it is
strictly dependent on the size of the interpolation $x$-grid and the amount of
target $\mu_F^2$ values.

For an average sized $x$-grid of $50$ points, and a single target $\mu_F^2$ the
size of the object in memory is of $\sim \SI{7.5}{MB}$, which scales linearly
with the amount of $\mu_F^2$ requested.
The dependency on the size of the $x$-grid is roughly quadratic.

\subsection{Storage}
\label{app:code:storage}

For permanent storage similar considerations applies with respect to the
memory object.
The main difference is that the object dumped by the \eko{} functions is always
compressed, leading to a size of $\sim \SI{500}{kB}$ for a single $\mu_F^2$,
which does not necessarily scales linearly with the amount $\mu_F^2$ since the
full rank $5$ tensor is compressed all-together (a linear scaling is just
the worst case).
Similar considerations applies to the dependency on the size of the $x$-grid.

\subsection{Possible Improvements}

There are a few easy directions to boost the current performances
of \eko{}, leveraging the $\mu_F^2$ splitting.

\paragraph{Jobs} To improve the speed of the computation, all the ingredients
of the final tensor (patches \& matching) can be computed by separate jobs, and
dispatched to different processors.
They just need to be joined at the very end in a simple linear algebra step.

Notice that the time measures presented in \cref{app:code:time} are obtained
with a fully sequential computation on a single processor, the only one
available at present time.

\paragraph{Memory} Since both the computation and the consumption of an \eko{}
can be done one $\mu_F^2$ at a time, it is possible to store each rank 4 tensor
on disk as soon as it is computed, and to load them in memory only while
applying them.\\

Both of these improvements are in the process of being implemented in \eko{}.
