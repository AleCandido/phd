In this paper we presented a new \qcd{} tool to perform perturbative \dglap{}
evolution.
In \cref{sec:theory} we reviewed some theory aspects that are relevant
for this paper. In \cref{sec:pheno} we presented a few applications
of the implemented \eko{} features.

Most of the work done to develop \eko{} has been devoted to reproduce known
results from other programs (and slightly extending or amending them to have a
consistent behavior), in order to have a more flexible framework where to
implement new essential features for physics study (more on this in the Outlook
at the end of this section).
Benchmarks with already existing and widely used codes are shown in
\cref{sec:pheno:bench}, and demonstrated to be successful.
Further, the multiple options and configurations available are presented in
subsequent sections and discussed, all leading to known and understood
behaviors.

This does not mean that the current status of \eko{} does not expose any
novelty. \cref{tab:comp} summarizes a general comparison on specific features
between several evolution programs; we list only tools targeting the same scope
of \eko{}, that is unpolarized \pdf{} fitting.
It is exactly for this target (\pdf{} fitting) that \eko{} is optimized, and
among the others three specific features are outstanding: the solution in
$N$-space, the backward \vfns{} evolution, and the operator-oriented nature.

\eko{} is the first code to solve \dglap{} in Mellin space that has been
explicitly designed to be used for PDF fitting, and while this may seem
irrelevant, it has been explicitly picked as an improvement for \eko{} over the
similar codes.
There are multiple solutions that are only available in $x$-space by applying
numerical approximated procedures, making the exact solution the most reliable
one. In $N$-space this is not required, and the choice of the solution is
left completely up to the user, with no numerical deterioration among the
alternatives (as it was already for \pegasus{}), and thus it can be based on
theory considerations.
Moreover, the perturbative \qcd{} ingredients used in the evolution (like
anomalous dimensions) are often first computed in $N$-space, making them
available for \eko{} immediately, while a further complex transformation is
needed for usage in the other codes.

All the programs listed are able to preform backward evolution in \ffns{}, that
consists in swapping the integral evolution bounds, but the \vfns{} backward
evolution is a unique feature of \eko{}, which involves the non-trivial
inversion of the matching matrix, as outlined in \cref{sec:theory:matching}.

The reason why \eko{} is an operator-first framework is discussed in detail in
\cref{app:code:motiv}, but essentially it makes \eko{} particularly suited for
our target: repeated evaluation of evolution for \pdf{} fitting.
Producing only operators makes \eko{} less competitive for single one-shot
applications, but the optimal scaling with the size of the task (practically
constant, since the time consumed is dominated by the operator calculation)
makes it extremely good for massive evolution (and already good when evolving
$\order{10}$ elements).

It should be observed that while the choice of Python as programming language
particularly stands out among the other programs (all written in Fortran,
either 77 or 95), this is only a benefit from the point of view of project
management (being Python much expressive) and third parties contributions
(since its syntax is familiar to many).
Indeed, we make sure not to experience Python infamous performances, when it
comes to the most demanding tasks (like complex kernels evaluation, or Mellin
inverse integration) as they either use compiled extensions (e.g.\ those available
in \href{https://scipy.org/}{\texttt{scipy}}~\cite{2020SciPy-NMeth}) or they are compiled Just In
Time (JIT), using the \href{https://numba.pydata.org/}{\texttt{numba}}~\cite{numba} package.


\renewcommand{\thefootnote}{\alph{footnote}}
\begin{table}
    \centering
    \begin{tabular}{l|llllll}
	Feature & \eko{} & \citelink{Bertone:2013vaa}{\apfel} & \citelink{Vogt:2004ns}{\pegasus} & \citelink{Salam:2008qg}{\hoppet} & \citelink{Botje:2010ay}{\qcdnum} \\
    \hline
    input space & $x$ & $x$ & $N,x^{*}$ & $x$ & $x$ \\
    solution space & $N$ & $x$ & $N$ & $x$ & $x$ \\
    delivery space & $x$ & $x$ & $N,x$ & $x$ & $x$ \\
    delivery & $\vb E$ & $\vb f$\footnotemark[1] & $\vb {\tilde f},\vb f$ & $\vb f$\footnotemark[1] & $\vb f$ \\
    backward \ffns{} & \checkmark & \checkmark & \checkmark  & \checkmark & \checkmark \\
    backward \vfns{} & \checkmark & & & (\checkmark)\footnotemark[2] \\
    intrinsic evolution & \checkmark \\
    \hline
    prog. language & Python & F77 & F77 & F95 & F77\\
    \lhapdf{} grids & \checkmark & \checkmark  \\
    interpolation grids & \checkmark & \checkmark
    \end{tabular}
    \begin{tabular}{cc}
		F77 = Fortan 77 & F95 = Fortran 95
    \end{tabular}
	\vspace*{5pt}
    \caption{Comparison between several evolution programs.
    The upper part of refers to some physical features: 
    by $x$ we mean the momentum fraction, $N$ the Mellin variables,
    $x^{*}$ denotes that \pegasus{} is able to deal with $x$-space input, 
    but only for fixed \pdf{} parametrization (see \cite{Vogt:2004ns}).
    $\vb E$ and $\vb f$ stands for evolution operators and \pdfs 
    respectively. 
    The lower part refers to program aspects, such as program language
    and interface with \lhapdf{}.
    }
    \label{tab:comp}
\end{table}
\footnotetext[1]{
    Both, \apfel{} and \hoppet{}, have an interface to access an evolution
    operator, but no one of the two can be used to store it and reuse it later on
    (this would require a dedicated interface).
}
\footnotetext[2]{
    \hoppet{} is able by default to do backward \vfns{}, but is not
    implementing intrinsic matching conditions (i.e.\ the contributions
    associated with the presence of an heavy flavor \pdf{}) nor the shifted
    matching scale.
}
\renewcommand{\thefootnote}{\arabic{footnote}}

While the main purpose of \eko{} is to evolve \pdfs, other \qcd{} ingredients
are required to perform the main task, like evolving the strong coupling
$\alpha_s$, running quark masses, or dealing with different flavor bases: they are
all provided to the end user, and for this reason actual results are shown in
this paper.

We remark that \eko{} is an open and living framework,
providing all ingredients as a public library, and accepting community
contributions, bug reports and feature requests, available through the public
\eko{} \href{https://github.com/N3PDF/eko/}{repository}.

\paragraph{Outlook} As outlined above \eko{} implements mostly well-known physics,
but we expect a series of upcoming project to build on the provided framework that will extend
the current knowledge on \pdfs.
Several features are already scheduled to be implemented,
and a few of them are already at an advanced stage:
the \nnnlo{} solution will be included as
soon as it becomes available~\cite{Moch:2021qrk}, while \nnnlo{} matching
conditions and strong coupling are already implemented and used in the
recent determination of the intrinsic charm content of the proton~\cite{Ball:2022qks}.

Another main goal of \eko{} is to provide a backbone in the determination of
\mhou{}, in the first place by allowing
the variation of the various scales used in the determination of evolved \pdfs,
that can be considered as an approximation to higher orders, implementing the
strategies described in \cite{AbdulKhalek:2019ihb}.
The variation of matching scales involved in \vfns{}
is already implemented and available.

Other planned features include: polarized evolution~\cite{Vogt:2008yw,Vogt:2014pha,Blumlein:2021ryt},
evolution of fragmentation functions~\cite{Mitov:2006ic,Moch:2007tx,Almasy:2011eq},
and the \qed$\otimes$\qcd{} evolution of
the photon-in-hadron distribution~\cite{Bertone:2017bme,Xie:2021equ,Cridge:2021pxm},
to estimate the impact of electro-weak corrections onto precision predictions. 
