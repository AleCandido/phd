% !TeX root = ../../../main.tex

As briefly introduced in \cref{sec:qcd/dglap}, a central element of a \pdf fit
consists in \dglap evolution.
%
During a fit, this will be used in two contexts: 
\begin{enumerate}[label=\roman*.]
  \item upgrading grids to the so called \fktab{}s (cf.\ \cref{ch:pine}), to
    obtain theory predictions at many different scales (matching the
    experimental results), starting from the \pdf candidate being minimized at
    a single fitting scale\footnote{
      Or, equivalently, evolving online (i.e.\ during the fit) at all those scales.
    }
  \item evolving the result of the fit, to distribute \pdf grids for the users
    already at all the potentially useful scales (interpolating among a finite
    sets of values)
\end{enumerate}

While working on new \pdf fits and related issues, a certain set of limitations
of the existing \dglap evolution libraries has been collected, and for this
reason we decided to write a new \qcd{} evolution library that matches the
requirements of a \pdf fitting collaboration, and similar fits as well.

We called it \acrfull{eko}, because the main focus is to compute such
operators, which are independent of the initial boundary conditions, and only
depend on the selected theory parameters.
In such manner, the operators can be computed only once, stored on disk and
then reused in the actual application. This method can lead to a significant
speed-up when \pdfs are repeatedly being evolved, as it is customary in \pdf{}
fits.
This approach has been introduced by
\fk~\cite{Ball:2008by,Ball:2010de,DelDebbio:2007ee} and it is further developed
here (with \fktab{}s cited above being the output of \fk).
%
Furthermore, we decided to solve the evolution equations~ in Mellin space (cf.\
\cref{sec:eko/theory-mellin}) to allow for simpler solution algorithms (both
iterative and approximated).
Yet, results are provided in momentum fraction space (cf.\
\cref{sec:eko/theory-interpolation}) to allow an easy interface with existing
codes.

\eko{} currently implements the leading order (\lo{}),
next-to-leading order (\nlo{}) and next-to-next-to-leading order (\nnlo{})
solutions~\cite{Vogt:2004mw,Moch:2004pa,Blumlein:2021enk}.
However, it is organized in such a way that the addition of higher order
corrections, such as the approximate next-to-next-to-next-to-leading order
(\nnnlo{})~\cite{Moch:2021qrk}, can be achieved with relative ease.
This accuracy is needed to match the precision in the determination of the
short-distance cross sections for several processes at LHC (see e.g.\
\cite{Duhr:2021vwj} and references therein).
We also expose the associated variations of the various scales (their role in
\pdfs is exposed in \cref{ch:mhou}, and specifically in \cref{sec:mhou/pdf}).

The correct treatment of intrinsic heavy quark distributions is also properly
taken into account.
While the role of these distributions in the evolution equations is
mathematically simple, as they decouple in a specific basis, their integration
into the full solution, including matching conditions, is non-trivial.
We implement backward evolution too, again including the non-trivial handling
of all matching conditions.
Both have been already used for studies of the heavy quark content of the
nucleon~\cite{Ball:2022qks}, that is exposed in details in \cref{ch:ic}.

It is also relevant to remark that \eko is another corner stone in a suite of
tools that aims to provide new handles to the theory predictions in the \pdf
fitting procedure.
This consists exactly in the calculation of those \fktab{}s mentioned above,
and described in \cref{ch:pine}.
But it is worth explicitly mentioning that \eko is also powering
\yadism~\cite{yadism}, the \dis coefficient function library described in
\cref{ch:dis}.

We adopted Python as the main development language, opting for a high-level
one, which is easy to understand and learn for newcomers.
In particular, with the advent of Data Science and Machine Learning, Python has
become the language of choice for many scientific applications, mainly driven
by the large availability of packages and frameworks, and in particular some
high-quality and widespread ones.
We decided to write a code that can be used by everyone who needs \qcd
evolution, and to make it possible for applications that are not supported yet
to be built on top of the shipped tools and ingredients.
For this reason the code is developed mainly as a library, that contains
physics, math, and algorithmic tools, such as those needed for managing or
storing the computed operators.
As an example we also expose a runner, making use of the built library to
deliver an actual evolution application. 
\newline

\eko is open-source, allowing easy interaction with users and developers.
The project comes with a clean, modular, and maintainable codebase that
guarantees easy inspection and ensures it is future-proof.
The repository is publicly available, and located at:
\begin{center}
\url{https://github.com/N3PDF/eko}
\end{center}
The \eko documentation contains the full \api reference, tutorials to get
started with \eko calculations, and far more complete discussion about the
underlying theory and numerical techniques adopted.
It can be accessed at:
\begin{center}
\url{https://eko.readthedocs.io/en/latest/}
\end{center}
This document is also regularly updated and extended upon the
implementation of new features.
