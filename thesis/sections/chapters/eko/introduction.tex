% !TeX root = ../../../main.tex

As we are entering the era of high-energy precision physics, theorists strive
to keep up with the experimental precision~\cite{Gao:2017yyd}.
The determination of parton distribution functions (\pdfs) is becoming
a major limiting factor and theory groups come up with more and more
involved procedures to improve the extraction~\cite{NNPDF:2017mvq,Hou:2019efy,Bailey:2020ooq}
eventually aiming for a one percent accuracy~\cite{NNPDF:2021njg}.
In order to achieve this goal a thorough treatment of theoretical
uncertainties is required~\cite{AbdulKhalek:2019ihb},
that so far was challenging with the current
state-of-the-art codes.
In this paper, we present \eko{} a new \qcd{} evolution library that matches
the requirements and desiderata of this new era.

\eko{} solves the evolution
equations~\cite{Altarelli:1977zs,Gribov:1972ri,Dokshitzer:1977sg} in Mellin
space (see \cref{sec:theory:mellin}) to allow for simpler solution algorithms
(both iterative and approximated).
Yet, it provides result in momentum fraction space (see
\cref{sec:theory:interpolation}) to allow an easy interface with existing
codes.

\eko{} computes evolution kernel operators (EKO) which are independent of
the initial boundary conditions but only depend on the given theory settings.
The operators can thus be computed once, stored on disk and then reused in the
actual application. This method can lead to a significant speed-up when \pdfs
are repeatedly being evolved, as it is customary in \pdf{} fits.
This approach has been introduced by \fk{}~\cite{Ball:2008by,Ball:2010de,DelDebbio:2007ee}
and it is further developed here.

\eko{} is open-source, allowing easy interaction with users
and developers.
The project comes with a clean, modular, and maintainable codebase that guarantees
easy inspection and ensures it is future-proof.
We provide both a user and a developer documentation. So, not only a user
manual, but even the internal documentation is published, with an effort to make
it as clear as possible.

\eko{} currently implements the leading order (\lo{}),
next-to-leading order (\nlo{}) and next-to-next-to-leading order (\nnlo{})
solutions~\cite{Vogt:2004mw,Moch:2004pa,Blumlein:2021enk}.
However, it is organized in such a way that the addition of higher order corrections,
such as the next-to-next-to-next-to-leading order (\nnnlo{})~\cite{Moch:2021qrk},
can be achieved with relative ease.
This accuracy is needed to match the precision in the determination of the
matrix elements for several processes at LHC (see e.g.\ \cite{Duhr:2021vwj} and
references therein).
We also expose the associated variations of the various scales.

\eko{} correctly treats intrinsic heavy quark distributions,
required for studies of the heavy quark content of the nucleon~\cite{Ball:2022qks}.
While the treatment of intrinsic distributions in the evolution equations is
mathematically simple, as they decouple in a specific basis, their integration
into the full solution, including matching conditions, is non-trivial.
We also implement backward evolution, again including the non-trivial
handling of matching conditions.

\eko{} is another corner stone in a suite of tools that aims to
provide new handles to the theory predictions in the \pdf{} fitting
procedure. To obtain the theory predictions in a typical fitting procedure,
first, one needs to evolve a candidate \pdf{} up to the process scale,
and then convolute it with the respective coefficient function.
The process specific coefficient function can be stored in the
\pineappl{} format~\cite{Carrazza_2020,christopher_schwan_2022_5846421}.
\eko{} is interfaced with \pineappl{} to produce
interpolation grids that can be directly used in a \pdf{} fit,
avoiding to do the evolution on the fly, but beforehand.
\eko{} is also powering \yadism{}~\cite{yadism} a new \dis{}
coefficient function library.

\eko{} adopts Python as a programming language opting for a high-level language
which is easy to understand for newcomers.
In particular, with the advent of Data Science and Machine Learning, Python has
become the language of choice for many scientific applications, mainly driven
by the large availability of packages and frameworks.
We decided to write a code that can be used by everyone who needs \qcd{}
evolution, and to make it possible for applications that are not supported yet
to be built on top of the provided tools and ingredients.
For this reason the code is developed mainly as a library, that contains
physics, math, and algorithmic tools, such as those needed for managing or
storing the computed operators. As an example we also expose a runner, making
use of the built library to deliver an actual evolution application. 
We apply modern best practices for software development, such as automated
tests, Continuous Integration (CI), and Continuous Deployment (CD), to ensure a
high quality of coding standard and a routinely checked code basis.

\paragraph{References} The open-source repository is available at:
\begin{center}
\url{https://github.com/N3PDF/eko}
\end{center}
In the following we do not attempt to give a complete overview over all
provided features and options, but limit ourselves to a brief review. The
full documentation instead can be accessed at:
\begin{center}
\url{https://eko.readthedocs.io/en/latest/}
\end{center}
This document is also regularly updated and extended upon the
implementation of new features.
