\documentclass[a4paper,11pt]{article}
% \pdfoutput=1 % if your are submitting a pdflatex (i.e. if you have
             % images in pdf, png or jpg format)

\usepackage{../styles/jheppub} % for details on the use of the package, please
                     % see the JHEP-author-manual

\usepackage[T1]{fontenc} % if needed
\usepackage{lmodern}

\usepackage{../styles/main}
\usepackage{../styles/defs}


\usepackage{microtype,xparse,tcolorbox}
\newenvironment{reviewer-comment }{}{}
\tcbuselibrary{skins}
\tcolorboxenvironment{reviewer-comment }{empty,
  left = 1em, top = 1ex, bottom = 1ex,
  borderline west = {2pt} {0pt} {black!20},
}
\newcounter{comment}[section]
\ExplSyntaxOn
\NewDocumentEnvironment {response} { +m O{black!20} } {\refstepcounter{comment}
  \IfValueT {#1} {
    \begin{reviewer-comment~}
      \noindent
      \textbf{Comment\hspace{1mm}\thecomment :\hspace{2mm}} \ttfamily #1 
    \end{reviewer-comment~}
  }
  \par\noindent\ignorespaces
} { \bigskip\par\par }
\ExplSyntaxOff


\begin{document} 
%\maketitle
%\flushbottom


We thank the referee for reviewing our manuscript once more, 
providing suggestions / comments for improvement.

In the following, we arrange the referee's comments into a series of items and 
will address them one by one.

\begin{response}{
  I have run the new example provided in the paper. I get $x*g(x)= 1.32630e3$
  at $x=10^{-7}$ and $Q=100$~GeV. This result differs slightly from the
  benchmark value $1.3272e3$. Am I supposed to get something close to, and
  compatible with, the benchmark value (but within what uncertainty?), or
  exactly that value? The authors do not say precisely. They should certainly
  make it clear what value one should get when running the code. If I am not
  getting the expected value, then something is wrong somewhere and deserves
  investigation.
}
The referee is correct pointing out that we were missing to report the
expected output value. Indeed, here is quite relevant, since the user want to
be sure to be running with the correct options and version. For this reason, we
are now providing the expected output in the comment in the listing.

The number found by the referee is the correct one, and it corresponds to an
error slightly smaller than 1\textperthousand{} which is compatible with
Fig.~1.
The main shift from the benchmark is probably coming from interpolation itself
(absent in the input \pdf, since it is not provided by \lhapdf, but present in
the \eko, and thus in the output).
\end{response}

\begin{response}{
  The authors spend quite some time in their resubmission letter explaining
  that they have carefully tested their code against existing ones, and that
  their code is a library and not a program. Please understand that I am not
  disputing this, and I am sure that they have done a very careful and thorough
  job. What I am complaining about, instead, is the usability of their library
  for an external user. Even after the last round of improvements, it is my
  belief that such usability remains poor.
}
We are also concerned about poor usability: we are going to improve it in our
library, and through the documentation.
For sure, something has been already introduced during the review process,
since - thanks to the referee - we are now providing tutorials to get started. 
We acknowledge them to be still not ideal, and most likely they might not cover
all the most common use cases. We will try to collect as many feedback as
possible (through the online repository, and related channels) in order to
continuously improve.
\end{response}

\begin{response}{
  I will note the following:

  \begin{itemize}
    \item Out of curiosity I installed the version of EKO that pip is providing
      by default (which happened to be the 0.10.2). The example given in the
      paper does not run with this version (KeyError: 'order').
  \end{itemize}
}
Indeed, this is due to an ongoing parallel development which is beyond the
scope of this paper.
This is only a problem in the input, but noticing this we decided to review the
user interface, and finally stabilize it.
As mentioned in the manual appendix we will keep the online tutorials 
up to date in order to provide consistent examples.
\end{response}

\begin{response}{
  \begin{itemize}
    \item Version 0.8.5 (the one suggested in the paper) does not seem to be able
      to run the example with LHAPDF (6.5.1) and CT14llo given in the online
      documentation. (ModuleNotFoundError: No module named 'ekobox')
  \end{itemize}
}
We are sorry for the inconvenience, but this is not a problem of compatibility
with \lhapdf: it is for sure originated from the multiple installations of \eko.
Indeed, \href{https://github.com/N3PDF/eko/tree/master/src}{\texttt{ekobox}} is
just a package inside the \eko{} distribution,
\href{https://github.com/N3PDF/eko/tree/v0.8.5/src}{missing in the
\texttt{0.8.5} release}, but present in \texttt{0.10.2}.

We are happy to help with this kind of issues, through the bug tracker on the
repository.
\end{response}

\begin{response}{
  All this is potentially confusing. Are the authors breaking backward compatibility
  even before a paper with an example is published? If they insist on 0.8.5, they
  should at least warn users that the example given in the paper cannot run on more
  recent versions (and provide an equivalent minimal, quick-start, example in the
  online documentation), and probably say that 0.8.5 cannot run the examples in the
  online documentation.
}
In this case, we want to remind that the documentation is actually versioned
together with the code, so in the optimal case the current status of the online
docs should refer to the latest version, that is not any longer \texttt{v0.8.5}.

In theory, this means that we could improve over the documentation of
\texttt{v0.8.5}, continuing the \texttt{v0.8.x} series, in which we only update
the documentation and keep the same code.
We believe that the proposed version is fully working, and able to do the task
we are talking about, so we would not need to improve the code in parallel, but
just the docs.
In practice, we do not have (m)any user(s), so we are trying to learn from the
shortcomings of this version to improve the future ones.

You can consider \texttt{v0.8.x} and \texttt{v0.10.y} as being two completely
different programs, so they can follow an independent life-cycle (that is what
a major change is in the end). Simply, it is not much useful at the moment to
insist on the \texttt{v0.8.x} program. But the moment it would be, we could
resume the development from that tag.
For this reason, we accept the critic about the lack of clarity, and we
published online the
\href{https://eko.readthedocs.io/en/v0.8.5}{\texttt{v0.8.5} version of the
docs}, we will point out the difference in the newest version of the docs.
Being a fairly common platform with a uniform interface, we believe potential
users to be familiar with the concept.
\end{response}

\begin{response}{
  I think that this paper is a borderline case, since while not providing
  original physics insight it describe nonetheless careful and useful work, but
  it fails to some extent to convey appropriate information to readers. Since a
  lot of time has already been spent by everyone involved on iterating this
  paper, and the marginal utility of continuing these iterations is clearly
  approaching zero, I recommend its publication at this stage.

  I have also a few words of advice for the authors.

  \begin{itemize}
    \item In order to give a wider user base access to their (preliminary)
      work, one does not need a published paper. An arXiv preprint is perfectly
      appropriate for this goal, and it can actually better evolve following
      the development of the code, and a paper can be published as soon as a
      stable version is reached.
  \end{itemize}
}
As mentioned before we believe that \eko{} \texttt{0.8.5} is complete enough to reproduce 
the existing codes with its own original strategy.
We thank the referee for this suggestion and we will consider this option
for future works.

\end{response}

\begin{response}{
  \begin{itemize}
    \item There are many packages that make a good or at least decent job of
      providing useful usage examples, be it in a paper and/or an online
      documentation and/or with the distributed package. The authors may for
      instance consult, among many others, Pythia, FastJet, or even LHAPDF.
  \end{itemize}
}
We don't feel the comparison with the mentioned programs is really fair
since they exist for a significant longer time (Pythia more then 40 years!)
and they target a different audience (specifically mostly experimentalists)
so they could iterate much more. In fact, we hope \eko{} has a similar life cycle
and we are sure our documentation will improve along the way. Instead, we rather
prefer to compare our documentation with those of similar tools, such as
\apfel{} and \pegasus{}, from which we are not shying away.
\end{response}

\begin{response}{
  \begin{itemize}
    \item Even sticking to the viewpoint that ``EKO is designed as a library
      and not as a program'', this does not mean that meaningful and simple
      examples cannot be provided. For instance, even working exclusively with
      the evolution operator on its grid (and therefore foregoing access to an
      evolved PDF at any $x$ and $Q$ point), the example provided might have
      shown its departure from the initial conditions at $Q0$ for increasing
      $Q$, at different values of $x$ and at different perturbative orders. It
      is just an example among many, but it would have concretely shown a user
      how to access the content of the evolution operator, and allowed for an
      appreciation of the numerical results.
  \end{itemize}
}
As already mentioned, we will try to provide more and more examples with the
collected feedback.
We also remind that, indeed, the main advantage of \eko{} is to have access to
the evolution kernel operator and not the evolved PDF. However, the evolution
of PDFs is a very common task in the community and the main benchmark we have,
so it is a natural point of focus. 
In fact, in the actual use case, we mainly use the operator to produce \fk{}
tables.
\end{response}


% \bibliographystyle{../styles/JHEP}
% \bibliography{../bibliography/eko.bib,../bibliography/refs.bib}

% \listoffixmes

\end{document}
