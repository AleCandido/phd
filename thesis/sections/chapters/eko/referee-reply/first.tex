\documentclass[a4paper,11pt]{article}
\pdfoutput=1 % if your are submitting a pdflatex (i.e. if you have
             % images in pdf, png or jpg format)

\usepackage{../styles/jheppub} % for details on the use of the package, please
                     % see the JHEP-author-manual

\usepackage[T1]{fontenc} % if needed
\usepackage{lmodern}

\usepackage{../styles/main}
\usepackage{../styles/defs}


\usepackage{microtype,xparse,tcolorbox}
\newenvironment{reviewer-comment }{}{}
\tcbuselibrary{skins}
\tcolorboxenvironment{reviewer-comment }{empty,
  left = 1em, top = 1ex, bottom = 1ex,
  borderline west = {2pt} {0pt} {black!20},
}
\newcounter{comment}[section]
\ExplSyntaxOn
\NewDocumentEnvironment {response} { +m O{black!20} } {\refstepcounter{comment}
  \IfValueT {#1} {
    \begin{reviewer-comment~}
      \noindent
      \textbf{Comment\hspace{1mm}\thecomment :\hspace{2mm}} \ttfamily #1 
    \end{reviewer-comment~}
  }
  \par\noindent\ignorespaces
} { \bigskip\par\par }
\ExplSyntaxOff


\begin{document} 
%\maketitle
%\flushbottom

\begin{response}{
The authors present a new open-source library, written in Python, meant to
provide tools to evolve parton distribution functions.}

We thank the referee for reviewing our manuscript and providing suggestions / comments for improvement.

In the following, we arrange the referee's comments into a series of items and 
will address them one by one.

\end{response}



\begin{response}{
This work adds one more item to a long list of DGLAP evolution packages that are
publicly available. While variety, and choice of computer language, is nice to
have, one can wonder about the need of a physics paper to present a code that
does not seem to contain new physical content.}

Due to an unfortunate oversight on our side the first version of the paper got wrongly submitted with
the article type \enquote{Regular Article}, where instead we wanted to submit as
\enquote{Tools for Experiment and Theory/Scientific Notes}. This should be corrected with the current version.

Now, it is indeed true that the paper mostly does not include any new physics content as
\dglap{} evolution equations are known for a long time. Instead, the aim of the paper
is to make the community aware of a new software tool that is tailored to address the specific needs in a common
task in the determination of \pdf{}s. Moreover, while this paper does (mostly) not include
new physics, instead we expect future development based on this work to discuss
genuinely new physics. In the paper we mention explicitly: (a) the determination of
intrinsic charm in the proton, (b) the determination of missing higher order
uncertainties in \pdf{}, and (c) the determination of \nnnlo{} \pdf{}s. Each of these
items we consider a major improvement in the determination of \pdf{}s.
We added a sentence to the outlook in conclusions to stress this point.

\end{response}



\begin{response}{
In a sense, the paper seems to read more like an advertisement of the code and
its features, with physical content simply sketched. While there would indeed be
no need to present it in detail, since it is in principle well know, I am not
sure if the middle ground chosen here is the best choice, as it is neither a
complete physics description, nor a sufficient manual for the code
(and I am not sure that the accompanying web pages are either).}

As explained in the previous comment the paper should really be with the article type
\enquote{Tools for Experiment and Theory/Scientific Notes} and hence it is considered
to be an advertisement of the code. We agree that the underlying physics is well known
and this one of the reason, why we refrained from repeating it here.
The other being, that indeed in the documentation we do spell out most details and formulas together with the suitable
references for the convenience of users and developers. As said in the paper we put
a dedicated effort for a good and extensive documentation. We added a paragraph at the beginning
of the theory section to highlight this more. We also added an explicit mention of the
API section in the documentation where we document \textit{all} available functions of the library.

\end{response}



\begin{response}{An example of the somewhat fuzzy description provided by the paper is the
beginning of section 3.3, where it is said that the internal Mellin
representation and the x-space output are ``bridged'' using a
Lagrange-Interpolation, but no further details are given, which is disappointing
as this is supposed to be one of the features of the new code (and one that has
an impact on the precision).}

We did expand section 2.2 and referred to it in section 3.3. We also added an explicit sentence
referring to Section 3.1 that interpolation is not blocking us in achieving the LHA benchmark.
We also want to point out that the full interpolation algorithm along with an example is spelled out
in the online documentation.

\end{response}



\begin{response}{Another weak point seems to be the performance of the code. The running times
mentiond in Table 2 seem to me to be way longer than those of other codes. I am
aware that python, as an interpreted language, is not expected to be very fast,
and that it can have other advantages. And the authors themselves mention that
their code may not be competitive under this point of view. However, at the very
least they should provide a more extensive and quantitative comparison with the
other packages currently widely used.}

The choice of the programming language is only a benefit for the overall
project management, since Python syntax is much expressive and widely known,
lowering the barrier for code inspection by third parties, and contributions as
well.

The moment we get performance intensive tasks, like integration, the usage of
external libraries with compiled extensions (like \texttt{scipy}, whose
\href{https://scipy.github.io/devdocs/tutorial/integrate.html}{\texttt{integration}}
sub-package relies on the \texttt{C++} implementation of \texttt{QUADPACK}
inside \texttt{GSL}) reduce the performance impact of the interpreted language
choice, since in practice it is only a driver for a program implemented in a
completely different language.

The second source of performance decoupling from Python, it is actually the
adoption of a Just In Time (JIT) compilation package (namely
\href{https://numba.pydata.org/}{\texttt{numba}}), that actually makes part of
the code being run not directly on the common runtime (the Python interpreter),
but compiled on \texttt{LLVM}-like back-end (the same back-end of
\texttt{clang} \texttt{C++} compiler, and many others) and off-loaded to a
stand-alone compiled function.
Effectively, this turns a restricted set of Python syntax into a separate
embedded language, in the sense that the encapsulated code undergoes a
completely different (compilation and) execution process than the rest of the
code base.

Only the availability of these tools has made the choice of Python possible for
this performance intensive task.
This is for sure not a novelty, since it is continuously happening in the
Machine Learning community, where the main libraries are all Python driven
(\texttt{TensorFlow}, \texttt{PyTorch}, \texttt{JAX}, \dots) and the task is
very performance demanding, and for the Quantum Computing community as well
(\texttt{Qiskit}, \texttt{Cirq}, \texttt{Q\#}, \dots), where the hardware is
potentially completely different from the usual CPUs.  That's why Python has
become common as a \enquote{glue language}.

Most likely this would result unclear to the user not familiar to the
implementation details, while the clear Python against Fortran comparison
suggests an actual performance loss.
Thus, both of these observations (compiled extensions and JIT) are now present
in the discussion of the comparison in Table 1.

Furthermore, we acknowledge that the former layout of Appendix A would result
in a direct comparison of \eko{} and \apfel{}, that we consider not to be
suitable.
For this reason, we added a whole new section at the beginning of the appendix,
introducing the optimal performance we have in mind for \eko{}, considering the
specific task it is being developed for.

When it comes to actually quoting times, we specified explicitly that the
difference is not only the \enquote{starting point} (time taken for the
elementary task), but even in the performance scaling with the number of object
being evolved.
Since the differences relies in the characteristics of a curve, there is no
unique way of telling which program \textit{is better}. Our goal was to
specifically support in the best way the evolution of a significant number of
border conditions ($\gtrsim 10$, but mostly $\order{100}$).

This is why we decided to keep not including an extensive time comparison,
since it would result in a rather complicated outcome.
Because of the dramatic difference in scaling it is clear when to use which
tool, and for the intermediate ground any tool is equally good (without the
need of a fine grained, benchmark dependent comparison).

\end{response}



\begin{response}{I am also wondering if Table 1 should be more detailed (besides better
explained). For instance, PEGASUS can return x-space distributions, and I think
HOPPET can do backward VFNS. C++ interfaces to at least some of these codes
probably exist (I seem to recall an APFEL++). It is also not clear to me what
the LHAPDF row entails.}

The referee is correct in stating that \pegasus{} can return evolved \pdf{}s directly in 
$x$-space. However, this is possible only for distribution with a fixed parametrization
(see section 3.3 of \cite{Vogt:2004ns}) and the program is not interfaced with \lhapdf{} 
directly, which makes the code hard to use in the context of \pdf{}s fitting.

We believe that \eko{} is the first program able to perform backward \vfns{}, while
with all the others codes (\apfel{}, \qcdnum{}, \pegasus{}, \hoppet{}) only backward \ffns{}
is allowed. Actually, the former is trivial and only requires to swap the
evolution integration bounds, but the latter is not because matching conditions at 
the heavy quark thresholds need to be inverted and properly joined together.

We are aware of the existence \apfelpp{}, but we decided not to include this program
in our comparison. \apfelpp{} is a C++ replacement of \apfel{}
that improves the code structure and maintainability, 
together with performance optimization.
However is currently developed targeting GPDs and TDMs fitting, being used both in 
\href{https://github.com/MapCollaboration/NangaParbat}{\texttt{NangaParbat}}. 
and \href{https://github.com/MapCollaboration/MontBlanc}{\texttt{MontBlanc}}
(see also the official \apfelpp{} \href{https://github.com/vbertone/apfelxx}{\texttt{Readme}}).
On contrary \apfel{} and \eko{} target only \pdf{}s evolution.
Furthermore \apfelpp{} has not yet reached the feature-equality with \apfel{}:
for instance \qed{} evolution, a flag feature of \apfel{}, is not implemented. 
Despite of being feasible, we don't expect that a comparison with \apfelpp{}
will add any new insight with respect to the one with \apfel{}.

In this respect the following actions have been taken. Updates on Table 1: 
\begin{itemize}
 \item We added a new row \enquote{input space} to explain the possibility of
 having $x$-space, or $N$-space \pdf{}s inputs.
 \item We added a new row \enquote{backward \ffns{}} to explain the possibility of
 performing \ffns{} backward evolution.
 \item We renamed the row \enquote{\lhapdf{}} to \enquote{produce \lhapdf{} grids}
 \item We added to the table and caption an explanation about the ability of \pegasus{} 
 of having inputs/delivery in $x$-space.
\end{itemize}
We recalled in the text body the difference between \ffns{} and \vfns{} backward evolution.
We mentioned in the text body the criterion used to select the tools in Table 1
\end{response}



\begin{response}{A few minor points further illustrate the need for at least a revision. Among
others, in the Introduction ``introspection'' may not be the word that the
authors were looking for, ``MHOU'' is first used on page 6 and then 13, but only
properly defined on page 17, and ``deserved'' on page 19 may again not be the
most appropriate choice of word.}

We swapped \enquote{introspection} with \enquote{inspection}.
We moved the definition of \mhou{} to page 4.
We swapped \enquote{deserved} with \enquote{required}.

\end{response}



\begin{response}{In conclusion, I am not sure to what extent this paper belongs in a physics
journal. At the very least, it should be extensively revised so as to ensure
that the information that is given, and that is original and specific to this
work, is complete and it is not drowned under more general, and fragmentary,
information. I think that the description of other codes, and the comparisons
with them, could also be reinforced, and made more quantitative at least in
terms of run times.}

As we said at the very beginning, the choice of the article type has
been an oversight on our side, since we really meant to write a paper to
present a new code, and we believe the most original contributions will come in
the future. This does not mean that there is no original physics implemented by
\eko{}. It is worth to stress the \eko{} novelties:
\begin{enumerate}
  \item It is the first program to work fully in $N$-space,
    addressing the need of a completely flexible $x$-space delivery
  \begin{itemize}
    \item $N$-space turns all convolutions in products, leaving an ordinary
      differential equation to be solved. In this way the problem is decoupled
      in two stages: this make it possible to deal with the 
      solution of the $Q^2$ differential equation in a much simpler way,
      and allows for a clean implementation of multiple solutions (like it has
      been done in \pegasus{})
    \item the pQCD ingredients (anomalous dimensions and operators matrix
      elements) are usually first available in $N$-space, making it possible
      for \eko{} to directly consuming them, without the need for an extra
      transformation step before
  \end{itemize}
  \item We believe backward \vfns{} evolution has been first implemented in \eko{}, including the
    contributions of intrinsic heavy quarks, that allows for the determination
    of the \pdf{} in any flavor number scheme, regardless of the stability of
    the perturbative matching (since it is compensated by the adaptive fitting
    process)
  \item We focus on an optimized operators delivery, that, as already argued, comes with great
    benefit for the \pdf{} fitting procedure
\end{enumerate}

It has been previously explained that a quantitative comparisons on run times
would not be much meaningful, while we agreed on the general comparison remark,
that's why we extensively expanded the discussion of Table 1 in the
conclusions, and the table itself.

\end{response}



\bibliographystyle{../styles/JHEP}
\bibliography{../bibliography/eko.bib,../bibliography/refs.bib}

\listoffixmes

\end{document}
