\documentclass[a4paper,11pt]{article}
\pdfoutput=1 % if your are submitting a pdflatex (i.e. if you have
             % images in pdf, png or jpg format)

\usepackage{../styles/jheppub} % for details on the use of the package, please
                     % see the JHEP-author-manual

\usepackage[T1]{fontenc} % if needed
\usepackage{lmodern}

\usepackage{../styles/main}
\usepackage{../styles/defs}


\usepackage{microtype,xparse,tcolorbox}
\newenvironment{reviewer-comment }{}{}
\tcbuselibrary{skins}
\tcolorboxenvironment{reviewer-comment }{empty,
  left = 1em, top = 1ex, bottom = 1ex,
  borderline west = {2pt} {0pt} {black!20},
}
\newcounter{comment}[section]
\ExplSyntaxOn
\NewDocumentEnvironment {response} { +m O{black!20} } {\refstepcounter{comment}
  \IfValueT {#1} {
    \begin{reviewer-comment~}
      \noindent
      \textbf{Comment\hspace{1mm}\thecomment :\hspace{2mm}} \ttfamily #1 
    \end{reviewer-comment~}
  }
  \par\noindent\ignorespaces
} { \bigskip\par\par }
\ExplSyntaxOff


\begin{document} 
%\maketitle
%\flushbottom


\begin{response}{
  The authors have addressed many of the points that I had raised in my first
  report and have upgraded their manuscript accordingly. The changes go in the
  right direction, and the information that the paper is being submitted to the
  "Tools" section certainly makes it more acceptable.
}

We thank the referee for reviewing our manuscript and providing suggestions / comments for improvement.

In the following, we arrange the referee's comments into a series of items and 
will address them one by one.
\end{response}

\begin{response}{
  I am still not entirely sure about statements like ``EKO is the first
  production-ready code to solve DGLAP in Mellin space'', which the authors may
  wish to explain better, as I think that other codes use Mellin Space.
}

The statements itself is valid, to the best of our knowledge, since the other
two DGLAP codes in Mellin space,
\href{https://www.nikhef.nl/~avogt/pegasus.html}{\pegasus{}} and
\href{https://github.com/vbertone/MELA}{\mela{}}, have never been used for PDF
fitting.

Nevertheless, we acknowledge that the \textit{production-ready} statement is
too generic and misleading, so we replaced with a more specific one.
\end{response}

\begin{response}{
  The big remaining issue, as I see it, is the accessibility and usability of
  the code. The paper, and the reply to the referee, contain sentences that
  make it abundantly clear that at least some of the authors are real experts
  in computing and python. However, the paper doesn't explain at all how to use
  the code. It refers, for documentation,  to a web site where, however, it's
  equally impossible to find examples to help one  get started (or even just a
  "Hello world!" example.

  In fact, using the instructions I could install the code using pip (which, by
  itself, may not be 100\% obvious for the average physicist like myself), and
  the github route is even harder, given the needed separate installation of
  poetry). However, after installation I could not find any clear instruction
  about how to proceed and how to test and use the code. The page
  https://eko.readthedocs.io/en/latest/overview/examples.html appears to be
  empty.

  A paper, and the documentation of a code, are meant to communicate to other
  people useful information. I believe that very relevant things are missing
  here. Since this work does not introduce new physics, it should at least
  properly present the code and show how to use it. As it is, this paper
  represents an advertisement of a new code but fails to make it accessible and
  therefore useful.
}

The referee is correct in raising this point, that we recognize as a weak point
of the submitted paper, and the \eko{} package itself.

For this reason, we added one more appendix, with a very brief guide to user
installation and usage, and paired it with a more detailed work-through in the
online documentation (in the section \enquote{Getting Started}).

This is not yet the best possible status of the User Manual, but we plan to
improve in the near future, leveraging on comments and any kind of feedback
received from our early users.
The current status should be enough to allow a generic user to install the
package, run a DGLAP calculation, and make use of the given result.
\end{response}

\begin{response}{
  Table 1 states that only the EKO code can carry out backwards VFNS. HOPPET is
  also able to do this as standard, to the same perturbative accuracy as for
  forwards VFNS.  Furthermore, HOPPET also has the option of calculating
  interpolation grids (though the public version of the code does not provide
  functionality for writing them out) and evolution operators.
}

We were genuinely not aware about these two features of \hoppet{}, mostly
because we are still lacking a careful comparison with it (as we did for the
other codes).

The ability to produce evolution operators is certainly relevant, but we
believe it to be comparable with the analogue \apfel{} feature, so it would
require a dedicated interface to consume it in a PDF fit.

On the other hand, the backward \vfns{} it is lacking proper support to
intrinsic flavors, that would lead to approximated results when PDF are
determined with a higher number of active flavors.
Moreover, while in the documentation this feature is indirectly mentioned at
the end of Section 6.2.1, it is never properly discussed, and that is the
main reason why we were not aware in the first place.

In any case, they are both relevant features of \hoppet{}, and we are now
acknowledging both of them in Table 1, while briefly mentioning in the
footnotes the differences with respect to those provided in \eko{}.
\end{response}

% \bibliographystyle{../styles/JHEP}
% \bibliography{../bibliography/eko.bib,../bibliography/refs.bib}

% \listoffixmes

\end{document}
