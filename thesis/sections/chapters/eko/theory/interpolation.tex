Mellin space has the theoretical advantage that the analytical solution of the
equations becomes simpler, but the practical disadvantage that it requires
\pdfs in Mellin space. 
This constraint is in practice a serious limitation since most matrix element
generators~\cite{Buckley:2011ms} as well as the various generated coefficient
function grids (e.g.\ \pineappl{}~\cite{Carrazza_2020,christopher_schwan_2022_5846421},
\appl{}~\cite{Carli:2010rw} and \fastnlo{}~\cite{Britzger:2012bs}) are not
using Mellin space, but rather $x$-space.

This is bypassed in \pegasus{} by parametrizing the initial boundary condition
with up to six parameters in terms of the Euler beta function.
However, this is not sufficiently flexible to accommodate more complex analytic
forms, or even parametrizations in form of neural networks.

We are bypassing this limitation by introducing a Lagrange-interpolation~\cite{LagrangeInterpol,suli2003introduction} of the
\pdfs in $x$-space on arbitrarily user-chosen grids $\mathbb G$:
\begin{equation}
    f(x) \sim \bar f(x) = \sum_{j} f(x_j) p_j(x),  \quad \text{with}\,x_j\in \mathbb G
\end{equation}
For the usage inside the library we do an analytic Mellin transformation of the polynomials $\tilde p_j(N) = \Md\qty[p_j(x)](N)$.
For the interpolation polynomials $p_j$ we are choosing a subset with $N_{degree} + 1$ points of the interpolation grid $\mathbb G$
to avoid Runge's phenomenon~\cite{zbMATH02662492,suli2003introduction} and to avoid large cancellation in the Mellin transform.
