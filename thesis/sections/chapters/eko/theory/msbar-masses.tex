In the context of \pdf{} evolution, the most used treatment of heavy quarks masses are the pole masses,
where the physical values are specified as input and do not depend on any scale.
However for specific applications, such as the determination of \mhou{} due to heavy quarks contribution 
inside the proton~\cite{Ball:2016neh}, \msbar{} masses can also be used.
In particular, in~\cite{Alekhin:2010sv} it is found that higher order corrections on heavy quark production
in \dis{} are more stable upon scale variation when using the \msbar{} scheme.
\eko{} allows for this option as it is discussed briefly in the following paragraphs.

Whenever the initial condition for the mass is not given at a scale coinciding with
the mass itself (i.e.\ $\mu_{h,0} \neq m_{h,0}$, being $m_{h,0}$ the given initial condition
at the scale $\mu_{h,0}$),
\eko{} computes the scale at which the running mass $m_{h}(\mu_h^2)$ intersects
the identity function.
Thus for each heavy quark $h$ we solve:
%
\begin{equation}
    m_{\overline{MS},h}(m_h^2) = m_h
\end{equation}
The value $m_h(m_h)$ is then used as a reference to define the evolution thresholds.

The evolution of the \msbar{} mass is given by:
%
\begin{equation}
    m_{\overline{MS},h}(\mu_h^2) = m_{h,0} \exp\qty[ - \int\limits_{a_s(\mu_{h,0}^2)}^{a_s(\mu_h^2)} \frac{\gamma_m(a_s')}{\beta(a_s')} \dd{a_s'} ]
    \label{eq:eko/msbarsolution}
\end{equation}
%
with $\gamma_m(a_s)$ the \qcd{} anomalous mass dimension available up to 
\nnnlo{}~\cite{Vermaseren:1997fq,Schroder:2005hy,Chetyrkin:2005ia}.

Note that to solve \cref{eq:eko/msbarsolution} $a_s(\mu^2)$ must be evaluated in 
a \ffns{} until the threshold scales are known. Thus it is important
to start computing the \msbar{} masses of the quarks which are closer to the
the scale $\mu_{0}$ at which the initial reference value $a_s(\mu_{0}^2)$ is given. 

Furthermore, to find consistent solutions the boundary condition of the
\msbar{} masses must satisfy $m_h(\mu_h) \ge \mu_h$ for heavy quarks involving
a number of active flavors greater than the number of quark flavors $n_{f,0}$ at $\mu_{0}$, implying that we find
the intercept between the \rge{} and the identity in the forward direction ($m_{\overline{MS},h} \ge \mu_h$).
The opposite holds for scales related to fewer active flavors.
