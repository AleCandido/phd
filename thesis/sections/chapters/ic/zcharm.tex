% !TeX root = ../../../main.tex

Here we provide full details on our computation of  $Z$+charm
production and on the inclusion of the \lhcb data for this
process in the determination of the charm \pdf shown in
\cref{fig:ic/Zc}. 

\paragraph{Computational settings.}
%
Theoretical predictions for
the $Z$+charm measurements in the forward region 
by \lhcb~\cite{LHCb:2021stx} follow the 
 settings described in~\cite{Boettcher:2015sqn}.
%
$Z$+jet events at \nlo \qcd theory are generated for $\sqrt{s}= 13$ TeV  using the $Zj$ package of the
\textsc{\small POWHEG-BOX}~\cite{Alioli:2010xd}.
%
The parton-level events produced by \textsc{\small POWHEG}
are then interfaced to \textsc{\small Pythia8}~\cite{Sjostrand:2007gs}
with the Monash 2013 tune~\cite{Skands:2014pea} for showering,
hadronization, and simulation of the underlying event and multiple
parton interactions.
%
Long-lived hadrons, including charmed hadrons,
are assumed stable and not decayed.

Selection criteria on these particle-level events are imposed
to match the \lhcb acceptance~\cite{LHCb:2021stx}.
%
$Z$ bosons are reconstructed in the dimuon final state by
requiring $60~\textrm{ GeV}\le m_{\mu\mu} \le 120~\textrm{ GeV}$,
and
only events where these muons satisfy
    $p_T^\mu \ge 20~\textrm{ GeV}$ and $2.0 \le \eta_{\mu}\le 4.5$
    are retained.
%
Stable visible hadrons within the \lhcb acceptance of
$2.0 \le \eta \le 4.5$ are clustered with
the anti-$k_T$ algorithm with radius parameter
of $R=0.5$~\cite{Cacciari:2008gp}.
%
Only events with a hardest jet satisfying
  $ 20~\textrm{ GeV} \le p_T^\textrm{ jet} \le 100~\textrm{ GeV}$
and $2.2 \le \eta_\textrm{ jet}\le 4.2$ are retained.
%
Charm jets are defined as jets containing
a charmed hadron, specifically  jets satisfying
$\Delta R(j, c\textrm{-hadron})\le 0.5$ for a charmed
hadron with $p_T(c\textrm{-hadron})\ge 5~\textrm{ GeV}$.
%
Jets and muons are required to be separated
in rapidity and azimuthal angle, so
we require $\Delta R(j, \mu)\ge 0.5$.
%
The resulting events
are then binned in the $Z$ bosom rapidity $y_Z = y_{\mu \mu}$.

The physical observable measured by \lhcb is the ratio of the fraction of $Z$+jet
    events with and without a charm tag,
    \begin{equation}
    \label{eq:ic/Rcj}
        \mathcal{R}_j^c \equiv \frac{\sigma(pp\to Z+\textrm{ charm~ jet})}{\sigma(pp \to Z+\textrm{ jet})}=
         \frac{N(c\textrm{ -tag})}{ 
        N(\textrm{ jets})} \, .
    \end{equation}
 Here  $N(c\textrm{ -tag})$ and $N(\textrm{ jets})$ are, respectively, the number
    of charm-tagged and un-tagged jets, for a  $Z$ boson rapidity interval
    that satisfies the selection and acceptance criteria.
    %
    The denominator of \cref{eq:ic/Rcj} includes all jets, even those
    containing heavy hadrons.
    %
The charm tagging efficiency is already accounted for at the level
of the experimental measurement, so it is not required in the theory
simulations.

Predictions for \cref{eq:ic/Rcj} are produced using our default \pdf
determination (\nnpdfr{4.0} \nnlo), as well as the corresponding \pdf set
with perturbative charm (see \cref{sec:ic/consistency}).
%
We have
explicitly checked that our results are essentially independent of the
value of the charm mass.
%
We have evaluated MHOUs and \pdf uncertainties using the
output of the \textsc{\small POWHEG+Pythia8} calculations.
We have checked that MHOUs, evaluated with the standard
seven-point prescription, essentially cancel in the ratio
\cref{eq:ic/Rcj}. Note that 
this is not the case for  \pdf uncertainties, because the dominant
partonic subchannels in the numerator and denominator are not the same.

\paragraph{Inclusion of the \lhcb data.}

%%%%%%%%%%%%%%%%%%%%%%%%%%%%%%%%%%%%%%%%%%%%%%%%%%%%%%%%%%%%%%%%%%%%%%%%%%%%%%%
\begin{table}[h]
  \small
    \renewcommand{\arraystretch}{1.45}
\begin{tabularx}{\textwidth}{C{2.5cm}YYYY}
  \toprule
 \multirow{2}{*}{ $\chi^2/N_\textrm{ dat}$}  &   \multicolumn{2}{c}{ default charm}   &\multicolumn{2}{c}{perturbative charm} \\
                       &  $\rho_\textrm{ sys}=0$   & $\rho_\textrm{ sys}=1$ &  $\rho_\textrm{ sys}=0$ &   $\rho_\textrm{ sys}=1$ \\
  \midrule
 Prior        &  1.85   &  3.33      &   3.54  & 3.85      \\
 \midrule
 Reweighted   &  1.81   &  3.14      &   $-$   &  $-$     \\
\bottomrule
\end{tabularx}
\vspace{0.3cm}
\caption{\label{tab:ic/chi2_zcharm} The values of $\chi^2/N_\textrm{ dat}$
 for the \lhcb $Z$+charm data before (prior) and after (reweighted)
 their inclusion in the \pdf fit. Results are given for two 
 experimental correlation models, denoted as
 $\rho_\textrm{ sys}=0$ and $\rho_\textrm{ sys}=1$. We also report values
 before inclusion for the perturbative charm \pdfs.
}
\end{table}
%%%%%%%%%%%%%%%%%%%%%%%%%%%%%%%%%%%%%%%%%%%%%%%%%%%%%%%%%%%%%%%%%%%%%%%%%%%%%%%

We first compare the quality of the description of the \lhcb data
before their inclusion. In \cref{tab:ic/chi2_zcharm} we show the
values of $\chi^2/N_\textrm{ dat}$ for the \lhcb $Z$+charm data
both with default and perturbative charm.
%
Since the experimental covariance matrix is not available for the \lhcb
data we determine the $\chi^2$ values assuming two limiting scenarios
for the correlation of experimental systematic uncertainties.
%
Namely, 
we either add in quadrature statistical and systematic errors ($\rho_\textrm{ sys}=0$),
or alternatively we assume that the total systematic uncertainty
is fully correlated between $y_Z$ bins ($\rho_\textrm{ sys}=1$). Fit
quality is always significantly better in our default intrinsic charm
scenario than with perturbative charm.
%
As is clear from
\cref{fig:ic/Zc} (top left), the somewhat poor fit quality is mostly due to the first
rapidity bin, which is essentially uncorrelated to the amount of
intrinsic charm (see
\cref{fig:ic/Zc}, top right).

The \lhcb $Z$+charm data are then included in the \pdf determination
through
Bayesian reweighting~\cite{Ball:2010gb,Ball:2011gg}. The
$\chi^2/N_\textrm{ dat}$ values obtained using the \pdfs found after their
inclusion are given in
\cref{tab:ic/chi2_zcharm}.
%
They are computed by combining the \pdf and
experimental covariance matrix so both sources of uncertainty are
included --- as mentioned above, MHOUs are negligible.
The fit quality is seen to improve only
mildly, and the effective number of
replicas~\cite{Ball:2010gb,Ball:2011gg} after reweighting
is only moderately reduced, from the prior $N_\textrm{ rep}=100$ to $N_\textrm{
eff}=92$ or $N_\textrm{ eff}=84$ in the
$\rho_\textrm{ sys}=0$ and $\rho_\textrm{ sys}=1$ scenarios respectively.
%
This
demonstrates that the inclusion of the \lhcb $Z$+charm measurements  affects
the \pdfs only weakly.
This agrees with the results shown in  \cref{fig:ic/Zc}~(center) in
\cref{sec:ic/intro}, where it is seen that the inclusion of the \lhcb data has
essentially no impact on the shape of the charm \pdf, but it moderately reduces
its uncertainty in the region of the valence peak.
