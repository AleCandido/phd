Once all the elements in \cref{eq:mhou/prescr/shifts,eq:mhou/prescr/thcovmat}
are spelled out, we have a clear recipe on how to compute the covariance matrix
$S_{ij}$.

For this reason, we are going to exploit all the properties that are required
or desirable (advantageous), in order to limit the available degrees of
freedom: anything left, it has to be regarded as being part of the
\textit{prescription}.

The current degrees of freedom are:

\begin{enumerate}
    \item the choice of the $p + 1$ dimensional space $\mathcal{V}_{ij}$ of all
        the accounted variations ($p$ renormalization scales, $1$ factorization
        scale)
    \item the choice of normalization coefficients $c_i(\vec{\kappa}) \in \R$
        \footnote{
            Not all values of $\R$ make sense, but there is quite a wide range
            of interesting variations: $\N$ for repeated points, or $\Q^+$ for
            normalizations (possibly coming from repeated points), or $0$ for
            masking.
            At this level, we are just not excluding anything that has no
            special reason to be excluded.
        }
    \item the choice of the default value $\vec{\kappa}_0$
\end{enumerate}

The last element is trivial: it's going to be part of the prescription, but in
the following we will always write $\vec{\kappa}_0 = \vec{0}$ for definiteness
(it's simple to replace this in the final result with $\vec{\kappa}_0$ in any
case).

\paragraph{Extra scales} We know that the predictions for each data point only
depend on two scales: the common factorization scale, and the related
renormalization scale, but not the others.
For this reason, it makes no sense to pick the normalization for point $i$
dependent on the other scales, since it would introduce a dependency of the
shifts on those scales that was not present in the unnormalized shifts.
Thus:

\begin{equation}
    \label{eq:mhou/prescr/2dim-norm}
    c_i(\vec{\kappa}) \equiv c_i(\kappa_F, \kappa_{R,i})
\end{equation}

\paragraph{Per-pair space} Next, we claim that the space $\mathcal{V}_{ij}$ can
not actually depend on the element $ij$ of the covariance matrix been
constructed. Indeed this stems directly for the necessity to prove
\cref{eq:mhou/prescr/pos} that is done in the following way:

\begin{align}
    \sum_{i,j} v_i S_{ij} v_j &= \sum_{i,j} \sum_{\vec{\kappa} \in \mathcal{V}_{ij}} v_i \Delta_i(\vec{\kappa})\Delta_j(\vec{\kappa}) v_j  =\\
        &= \sum_{\vec{\kappa} \in \mathcal{V}} \sum_{i,j} v_i \Delta_i(\vec{\kappa})\Delta_j(\vec{\kappa}) v_j =\\
        &= \sum_{\vec{\kappa} \in \mathcal{V}} \left(\sum_{i} v_i \Delta_i(\vec{\kappa})\right)^2 > 0
\end{align}

If the space $\mathcal{V}$ were actually dependent on $ij$, it would have not
been possible to swap the two sums in the second step.

\paragraph{Space choice} On the other hand, it is desirable to define the
prescription only on the space of relevant scales for the given point $ij$.
This means the factorization scale $\kappa_F$ and

\begin{description}
    \item[off-diagonal] two renormalization scales $\kappa_{R,i}$ and
        $\kappa_{R,j}$, or
    \item[diagonal] even a single one, if the two points are related to the
        same process, i.e. 
        \begin{equation*}
            \kappa_{R,i} = \kappa_{R,j}
        \end{equation*}
\end{description} 

We would like our expressions not to depend on the number of scales present,
and only account for the scale relevant for the pair $ij$ being considered.
The easiest choice is to pick the space $\mathcal{V}$ to be fully factorized in
the various dimensions of $\vec{\kappa}$. This means that it can be written as
\begin{equation}
\mathcal{V} = \prod_{i = 1}^{p+1} v_{i},
\end{equation}
with $v_{i}$ the one-dimensional space representing the variation of the single
scale labeled with $i$. 
\newline

But this is not the only choice available, it is just the simplest.
There is only one more option that guarantees the independence of the
projection on the pair $ij$, i.e. factorize the space for each possible value
of $\kappa_F$.
This option will be explored in \cref{sec:mhou/prescr/slices}.
\newline

In the case of a fully factorized space, the complex choice of the space is
reduced on $p + 1$ choices for one dimensional spaces.
But if there is no reason to distinguish processes at this level, it is
reasonable to pick the same space for each renormalization scale.

In practice, the basic one dimensional space will be always the same\footnote{
    The one spelled out is only an option, any other space would work equally
    well.
}:
\begin{equation}
    \label{eq:mhou/prescr/1dim-space}
    v = \{-\log(2), 0, \log(2)\} \equiv \{-, 0, +\}
\end{equation}
and the overall space will be just the product:
\begin{equation}
    \mathcal{V} = v^{p + 1}
\end{equation}

\paragraph{Normalization} At this point, all the arbitrariness left for the
prescription is encoded in the normalization coefficients.
With our simple choice of the space there is no reason to choose complex
coefficients, thus we will define the following prescriptions:

\begin{equation}
\label{eq:mhou/prescr/norm_coeff}
    c_i(\vec{\kappa}) = 
    \begin{cases}
        1 / \sqrt{N_m}     \qquad &\kappa \in \mathcal{V}^i_m\\
        0                  \qquad &\text{else}
    \end{cases}
\end{equation}

The spaces $\mathcal{V}_m^i$ now defines our point prescription, together with
the overall normalization $N_m$, since the $c_{i}(\vec{\kappa})$ are acting as \textit{masks} on the points $\vec{\kappa}$ not belonging to the space.
For the former we'll choose:
\begin{equation}
    \mathcal{V}_m^i = v_m^i \times \{-, 0, +\}^{p-1}
\end{equation}
where the two dimensional spaces $v_m^i$ are always the same space $v_m$, but
for the scales $(\kappa_F, \kappa_{R,i})$, while the other scales are free to
assume any possible value.

For the normalizations instead, there is no strict nor reasonable way to fix it
completely, but it is possible to fix the scaling in the case of a space
$v_m$ and $v$ with an hypothetically large number of point: since we don't want
the normalization of the theory covariance matrix to depend on the number of
points being in the prescription, we'll choose
\begin{equation}
    \label{eq:mhou/prescr/N-norm}
    N_m \propto \abs{v_m} \cdot \abs{v} = m \cdot 3^{p-1}
\end{equation}
