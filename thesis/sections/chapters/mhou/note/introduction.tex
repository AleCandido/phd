% !TeX root = ../../../../main.tex

As introduced in \cref{sec:mhou/estimates}, in the \textit{scale variations}
approach the scales to be varied are actually two: the \textit{renormalization}
and the \textit{factorization} scales.

More precisely, only the kind of scales to be varied are two, but there is one
renormalization scale associated to each \textit{process}\footnote{
  Unfortunately, also the process definition is not unambiguous, being
  essentially a way to gather in groups experimental data point.
  An indication is given by the theory predictions associated, and the
  intuitive idea is that each group is identified by a different \lo Feynman
  diagram.
  Because of this, it might happen that some processes correspond to \nlo or
  further contributions to other simpler processes.
}, so the amount of scales is essentially $p + 1$, where $p$ is the amount of
processes in the dataset.
Thus, a consistent way of varying together all these scales is required in a
global \qcd fit, like those for collinear \pdfs, because the amount of
processes might quickly scale, and rough choices for prescriptions might result
in undesirable features for the theory covariance matrix generated.
Also consider that, while dealing with a large number of data points (even for
a small amount of processes), covariances become crucial, since they scale as
$\order{\nd^2}$, while the variances are just $\nd$.
The specific choices for a coordinated choices of scale values is called a
\textit{point prescriptions}, because consist in the selection of a finite set
of points (and related weights) in the space of possible values for the
unphysical scales, use to estimate the whole observable dependence.

In the following, we present the derivation of suitable point prescriptions,
that can be used in the construction of a positive semi-definite theory
covariance matrix.
It turns out that two classes of prescriptions are possible, both requiring
milder or stronger generalization of equations in \cite{NNPDF:2019ubu}.
The actual result claimed in the paper can be obtained within the broader
generalization, ensuring the positive semi-definiteness of the covariance
matrix computed with that class of prescriptions.

There are two conditions that we want to satisfy in constructing the theory
covariance matrix, in order to support the interpretation as the covariance
matrix of our theory prior distribution.

\begin{enumerate}[label=\Alph*.]
    \item We want the theory covariance to be \textbf{generated by some shift
        vectors} $\Delta_i(\vec{\kappa})$; the vectors should be proportional
        to the difference of predictions obtained by a theory variation
        $T_i(\vec{\kappa})$ and the default theory in which $\vec{\kappa} =
        \vec{\kappa}_0$

        \begin{align}
            \Delta_i(\vec{\kappa}) &= c_i(\vec{\kappa}) \left(T_i(\vec{\kappa}) - T_i(\vec{\kappa}_0)\right)
            \label{eq:shifts}\\
            S_{ij} &= \sum_{\vec{\kappa} \in \mathcal{V}_{ij}} \Delta_i(\vec{\kappa})\Delta_j(\vec{\kappa})
            \label{eq:thcovmat}
        \end{align}
    \item We want it to be \textbf{positive semi-definite}, as required for any
        covariance matrix

        \begin{equation}
            \label{eq:pos}
            v_i S_{ij} v_j > 0 \qquad \forall v \in \R^{\nd} 
        \end{equation}
\end{enumerate}
