In \cref{sec:mhou/prescr/deriv} we made a set choices for the degrees of
arbitrariness exposed at the beginning.
All of them were yield by a strict requirement (needed to obtain a property,
like $S_{ij} \ge 0$) or by a reasonable request (e.g.\ not adding further
dependencies with normalizations, which led to \cref{eq:2dim-norm}).
Only in one single case we made an assumption based on an unneeded simplicity:
the choice of the space as fully factorized.

This choice is sensible for the renormalization scales: why should the space
look different seen from the perspective of different processes? Why different
processes should be correlated by the space?
On the other hand, it is completely arbitrary for the factorization scale.
Since factorization scale $\kappa_F$ is treated separately from renormalization
scales $\kappa_{R,i}$, no surprise if even the space symmetry somehow is broken
on $\kappa_F$\footnote{
    For the $\kappa_{R,i}$, choosing them factorized and uniform as argued, a
    permutation invariance is present, and makes sense.
    No reason to extend it to $\kappa_F$.
}.

Thus, we can have a different factorized space for each different value of $\kappa_F$:
\begin{align}
    \label{eq:sliced-space}
    \mathcal{V} &= \bigsqcup_{\kappa_F \in v_F} \mathcal{V}(\kappa_F)\\
    \label{eq:1f-factorized}
    \mathcal{V}(\kappa_F) &\equiv v(\kappa_F)^p
\end{align}
where $v_F$ is the space of possible values of $\kappa_F$ (usually it will be
just $v$ of \cref{eq:1dim-space}), and $v(\kappa_F)$ is instead the space of
renormalization scales related to that single value of the factorization scale.

In this case also the definition of the normalizations $c_i(\vec{\kappa})$
should change with respect to those defined in \cref{eq:norm_coeff} in order to
account for this, since the different spaces contain different numbers of
points.
We decide to normalize the elements such that once the full space is projected
over each of the two dimensional spaces $(\kappa_F, \kappa_{R,i})$, the
coefficients of the various shifts are equal to one, thus:
\begin{equation}
    \label{eq:sliced-norm}
    c_i(\vec{\kappa})^2 \propto \frac{1}{\sum_{\kappa_F'}v(\kappa_F')} \frac{\abs{v(\kappa_F)}}{\abs{\mathcal{V}(\kappa_F)}}
        = \frac{1}{m \cdot \abs{v(\kappa_F)}^{p-1}}
\end{equation}
since the scales projected are all renormalization scales but a single one,
that is the relevant one for the given $i$, and together with $\kappa_F$ make
the two dimensional space, whose volume is $\sum_{\kappa_F'}v(\kappa_F') = m$.
