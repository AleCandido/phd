% !TeX root = ../../../main.tex

Components have applications on their own, and part of them have already been
used (or are being used) to support other works.
Even though here it appears incidental, this is an important design feature: we
are building a framework, not just a pipeline application. The
various components should be focused on dedicated tasks and easy to integrate
in different architecture (or, more realistically, stand-alone projects), for
similar but different goals.

A first example is the study on evidence
for an intrinsic charm component in the proton \cite{Ball:2022qks}, based on
the \nnpdf 4.0 \pdf set, latest release of the \nnpdf family, and \eko
\cite{Candido:2022tld}, the evolution code described in the previous
\cref{sec:pine/arch}.
The role of \eko has been to unfold the intrinsic component from the so-called
fitted charm, in the 4 flavor number scheme (default scheme at fitting scale
for \nnpdf), by backward evolving with \dglap equation in a 3 flavor number
scheme \pdf set at a lower scale.
On top of the required backward evolution, and the proper treatment of
intrinsic components,
\eko implemented the \nnnlo matching conditions
between the 4 and 3 flavor schemes, that have been relevant to estimate the
perturbative stability of the result obtained.

Another application is the study of the forward backward asymmetry in the
Drell--Yan process with a high cut in the invariant mass of the
lepton pair \cite{Ball:2022qtp}.
In particular, the work focuses on the comparison between results obtained with
the \nnpdf 4.0 \pdf set and other contemporary \pdf sets from different
collaborations. We find that a certain shape in the high cut setting is
related to the specific shape of the \pdfs in the large-$x$ extrapolation
region, and so very sensitive to the possible bias of extending behaviors
typical of the central data region.
In this context, it has been crucial to have \pineappl \cite{Carrazza:2020gss}
\cite{christopher_schwan_2022_7145377} grids pre-computed to reproduce the
results, iterating on the \pdf set to investigate different features of the
\pdf, and trying to trace back the distribution behavior to \pdf features.

Finally, a study of the low energy neutrino structure functions is ongoing,
where the low $Q^2$ experimental data is reconciled to the known
perturbative calculation at higher energies, based on the \pdfs.
Here, we use \yadism, a general inclusive DIS provider
interfaced with \pineappl, to produce perturbative \qcd calculation for the
structure functions that get matched to experimental data.
