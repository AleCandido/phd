% !TeX root = ../../../main.tex

Performing a \pdf fit requires to integrate several elements to be gathered
from many different sources: \textit{data} from several experiments, ranging
over multiple decades and formats, and competitive \textbf{theory predictions},
coming from different providers.
Finally, a fitting methodology has to be selected and engineered to implement
theory constraints, and to limit not physically motivated bias.

While data are a \textit{static} component in the fit, the theory predictions
depend on the candidate \pdf, since they are the map that connect the
unobserved \pdf space, to the observed data space.
During the fit, this map will be used a large number of times (at least once
for every minimization step), so it is paramount to have an efficient way to
evaluate it, otherwise it can become a serious bottleneck.

For this reason, a few interfaces to \pdf independent theory predictions have
already been implemented
\cite{Carli:2010rw,Britzger:2012bs,Britzger:2022lbf,Carrazza:2020gss}, and they
are used in different contexts.
They propose different file formats to store the output of a Monte Carlo
generator, splitting them by luminosity component, perturbative order, and
observables binning.
This output can be optimized as an \textit{interpolation grid}, leveraging the
fact that the \pdf itself is only defined over a finite set of points, and thus
including the interpolation basis in the factorized cross-section.
Essentially, this recast the partonic cross sections predictions as
a \textit{theory array}, for which the Mellin convolution is replaced by a
linear algebra contraction over a single or multiple \pdf set.
This idea can be broadened to apply to any factorized function, describing the
structure of an external hadron (both incoming and outgoing).

However, this picture does not exhaust the needs of a \pdf fit (or any other
hadronic one), because, while the \pdf dependence on flavor and $x$ value is
folded on the grid, that on factorization scale has to be fixed to the process
dependent value.
This dependence is not fitted, since it is only determined by perturbative
\qcd.
In order to obtain it, it is required to solve the \dglap equation with the
border condition provided by the fit.
But being \dglap a set of integro-differential equations \textit{linear} in the
\pdf, this can be converted in the application of a suitable \textit{evolution
operator}, solving the same equation, as discussed in
\cref{sec:qcd/dglap,ch:eko}.
Since the evolution operator can also be computed ahead of time, it is possible
to combine the two ingredients (the operator and the grid) in a single fast
array interface, that will directly produce the required theory predictions
once contracted with the \pdf candidate.
Thus, the map from \pdf space to data space discussed above, is reduced to a
linear algebra product (or more than one, when multiple hadrons are involved).
During this operation, there is no loss of generality, since the interpolation
basis used for the conversion of the analytic convolutions is already present
in many \pdf applications due to their non-perturbative nature.
Such an interface is called a \enquote{Fast Kernel table} (shortened to
\textit{FK table}) in the context of the \nnpdf collaboration.

To produce the \textit{\fktab{}s} an evolution operator provider is required,
and needs to be interfaced with the grids. This was originally done in \nnpdf
by an internal tool (\fkgen), and then systematized in the \apfelgrid
\cite{Bertone:2016lga} package (leveraging the \apfel~\cite{Bertone:2013vaa}
evolution library), later reworked once more taking the name of \apfelcomb
\cite{APFELcomb}.

An array interface is extremely useful, since it allows to treat the theory map
in the context of many software frameworks, just relying on the data structure
of an array. Especially relevant for machine
learning software frameworks, but not limited to them, e.g.\ it allows to
create a Bayesian inference-based methodology (cf.\ \cref{ch:gp}), without the
need of the treatment of further complex functions.
