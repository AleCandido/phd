% !TeX root = ../../../main.tex

%************************************************
\section{Final Remarks}
\label{sec:pos/conc}
%************************************************

The goal of this work has been the construction of a universal factorization
scheme in which PDFs are non-negative.
In order to attack the problem, we started from the observation that \msbar{}
partonic cross sections for typical electro- and hadro-production processes are
not positive. This then implies that positivity of the PDFs is not guaranteed,
since folding a negative partonic cross section with a positive PDF could lead
to a negative physical cross section.
We have then traced negative partonic cross sections to the way collinear
subtraction is performed in \msbar{} and specifically we have shown that it is
due to over-subtraction, related to the choice of subtraction scale, and also
the treatment of the average over gluon polarizations in $d$ dimensions.
This loss of positivity only manifests itself  in off-diagonal quark-gluon and
gluon-quark channels.

A universal subtraction prescription  which preserves positivity of the partonic
cross section can then be constructed using hadronic kinematics, and
shown to preserve positivity also in electroproduction kinematics. This
prescription does not automatically respect momentum conservation,
which however can be enforced with a soft modification of the subtraction
procedure that does not affect its positivity properties. By
performing collinear factorization in the standard approach of
Refs.~\cite{Collins:1981uw,Curci:1980uw} it is then possible to show
that positivity of the PDFs, defined  as probability
distributions, is preserved at all stages, so PDFs remain positive.

In fact, this positivity is a manifestation of the fact that PDFs can
always be defined in terms of a physical process: what PDFs do is to
allow one to express the perturbative QCD prediction for a process in
terms of that for another process. The definition of the PDFs can then
be 
process-independent (as in \msbar{}) or process-dependent (as
in so-called physical schemes~\cite{Catani:1995ze,Diemoz:1987xu}).
Its positivity will then be preserved provided only that the renormalization
conditions, which fix the value of operator matrix elements that
define the PDFs, preserves their interpretation as moments of a
probability distribution. Effectively, this corresponds to choosing
positive Wilson coefficients.

By considering a scheme in which PDFs are manifestly positive, and the
transformation from it to \msbar{}, we have finally shown that in the
\msbar{} scheme PDFs remain positive, despite the fact that off-diagonal partonic
cross sections are negative. From a physical point of view, this is a consequence
of the fact that the \msbar{} subtraction is actually strongly positive
in the diagonal channels (where by ``strongly'' we mean that
partonic  functions tend to $+\infty$  towards kinematic
boundaries). This then overwhelms the negative contribution from
off-diagonal channels, while away from kinematic boundaries
off-diagonal channels are perturbatively subleading.

Positivity of the PDFs is neither necessary nor sufficient for physical
cross sections to be positive, as they ought to: it is not necessary,
because it is possible that a negative PDF still leads to a positive
hadronic cross section once folded with a suitable coefficient
function, and it is not sufficient because in a scheme, such as
\msbar{},
in which some partonic cross sections are negative it could well be
that, while the true PDF must necessarily lead to positive measurable
cross sections,  an incorrectly determined PDF could lead to a negative
cross section despite being positive.

In other words, it is not necessarily true that 
the region in PDF space which is excluded by the requirement of
positivity of the PDF is the same as that which is excluded by
requiring positivity of the cross sections. However, from the point of
view of PDFs determination, knowing that PDFs must be
positive in a given factorization scheme does provide a useful constraint,
in that it excludes a region which does not have to be explored,
though this restriction is not necessarily the most stringent one. It
is natural to ask whether the positivity requirement could be
more restrictive in some factorization schemes than others, but it is
unclear whether and how this question could be answered. The question
of optimizing the scheme choice from the point of view of positivity
constraints, for the sake of PDFs determination, remains open for
future investigation.

