% !TeX root = ../../../main.tex

\pdf{}s are a set of functions of two variables (i.e. a function of three
variables, of which one is discrete):
\begin{equation}
  f_i(z, \mu_F^2)
\end{equation}

The three variables are:
\begin{itemize}
  \item the \textit{flavor} $i$ of the chosen parton (usually a \pid in practice)
  \item the \textit{momentum fraction} $z \in (0,1]$ carried by the parton
  \item the \textit{factorization scale} $\mu_F^2 \in \mathbb{R}^+$
\end{itemize}
where the last one is required, since \pdf{}s are defined through the
factorization theorem, and the factorization scheme used usually involves an
unphysical scale, $\mu_F^2$, in a very similar way to what happens for
renormalization schemes \cite{Ellis:1996mzs}.

The role of the \pdf fits is to determine a border condition at a given value
for the unphysical scale, let it be $\mu_{F,0}^2$, since the dependence on the
scale is fully encoded in perturbative \qcd.
%
Indeed, the various schemes, corresponding to different choices of the
unphysical scale $\mu_F^2$, are related by the analogue of Callan-Symanzik
equations for factorization, obtained asserting that physical observables must
not depend on the choice of the unphysical scale.
%
In such manner, for choices of the scale in the perturbative regime for \qcd,
some terms are factorized either in the \pdf, or in the hard partonic cross
section.
Moving this scale, the terms are swapped, but thus the \pdf values in two
different schemes, corresponding to two different values of the scale, should
compensate for the difference in the partonic cross sections, both obtained by
a perturbative calculation, thus the difference is also determined by
perturbative physics.

This relation between \pdf{}s defined at different scales takes the shape of a
set of integro-differential equations, called the \acrfull{dglap}:
\begin{equation}
	\muF^2 \dv{\vb f}{\muF^2}{}(x,\muF^2) = \vb P (a_s(\muR^2),\muF^2) \otimes \vb f(\muF^2)
	\label{eq:qcd/dglap}
\end{equation}
The equations establish the anomalous scaling of the \pdf{}s, and the kernels
${\vb P}$ are called \textit{Altarelli--Parisi splitting functions}.

The equation and its solution is discussed further in \cref{ch:eko}, where
another software package is presented, \eko, automating the solution of the
associated operator equation.
%
Indeed, any linear equation (as \dglap) can be solved by a linear operator,
that is actually producing the solution given any boundary condition, thus
independently of the boundary condition itself:
\begin{equation}
  f_i(z, \mu_F^2) = E_{ij}(\mu_F^2 \gets \mu_{F,0}^2) \otimes f_j(\mu_{F,0}^2)
\end{equation}
We call such an operator, for \dglap, \acrfull{eko}.

This is another central ingredient in \pdf fits, since the data set span
different scales, over multiple order of magnitudes, so the \pdf determined by
the fit has to be evolved first to the suitable scale, to be folded with the
partonic calculation at that scale, resulting in the hadronic predictions for
the measured observable.

