\acrfull{qcd} is the theory of strong interactions among colored particles.
It is a fundamental constituent of the \acrfull{sm} of particle physics,
together with the \acrfull{ew} interaction.

The fundamental feature of \qcd is its asymptotic freedom, that makes the
coupling perturbative at high enough energies, but non-perturbative at low
energies, where the relevant scale to compare is the intrinsic
$\Lambda_{\text{QCD}}$, whose order of magnitude is roughly the same of the
mass of the proton ($M_p$).
%
This generates a large amount of composite particles, named \textit{hadrons},
whose constituents are colored particles, i.e.\ quarks, who are determining
main quantum numbers on the hadrons, and gluons, the interaction carrying boson
arising in the corresponding \acrfull{ym} theory, and acting as the binding
glue between quarks.

The discovery of hadrons beyond proton and neutron has driven particle physics
experiments advancements and theoretical progress in the second half of the
twentieth century, resulting in the successful framework of \qcd as a
\acrlong{qft} following the \ym pattern.
But other interesting theories has been investigated during the quest for a
theory of hadrons, and some of them still have relevant consequences, extending
beyond hadrons (e.g.\ string theory, or the old bootstrap).

Despite the success of the framework, the nature of hadrons is still being
intensively investigated, since it would require the solution of
non-perturbative \qcd dynamics.
%
Different tools are available for this investigation, one of them being the
extremely powerful formulation of \qcd on a discretized lattice
\cite{Wilson:1974sk}, but it has not yet been possible to describe the nature
of hadrons from first principles, by a sufficiently accurate lattice
determination.
%
But hadrons are ubiquitous in \hep experiments, both as products of high-energy
collisions, and as scattering particles at hadronic and semihadronic machines,
so a better understanding of the hadronic structure is required to formulate
precise enough theoretical predictions for collision events, confirming in this
way the \sm theory, and investigating possible signals of \acrfull{bsm}
physics.

To circumvent the current limitations about non-perturbative \qft, a different
approach has been pursued, whose origin dates even before the discovery of
\qcd.
In order to analyze collisions of composite particles, it was proposed to
describe them as a packet of collinear point-like constituents, collectively
named \textit{partons}, each one sharing a fraction of the measured momentum of
the scattering particle \cite{Feynman:1969wa}.
%
This partonic picture was successfully applied to predict the high-energy
electron-proton collisions, a process called \acrfull{dis}, resulting in the
discovery of Bjorken scaling \cite{Bjorken:1967fb}, i.e.\ the statement that
\dis structure functions (see \cref{sec:qcd/dis}) do not depend on the
exchanged photon virtuality, setting the energy scale of the process.

Bjorken scaling is then violated by \qcd corrections, but it is still possible
to remain in the framework established by the parton model, because of a
fundamental \qcd property: factorization \cite{Collins:1989gx}.
%
This feature ensures that, up to highly suppressed corrections in the scale
ratio, the hadronic cross-sections factorize in a perturbative hard partonic
cross-sections, that can be computed in perturbation theory with \pqft
calculations, and a universal matrix element, describing the probability that a
certain parton constituent from the original hadron enters the hard event.

It is in this framework that \acrfull{pdf} are defined, as the non-perturbative
matrix element, completing the hadronic cross-section.
\pdf universality ensured by factorization, i.e.\ being independent from the
process considered, makes possible to have a unique set of functions describing
the hadronic structure, bridging the gap from the partonic to the hadronic
cross-sections.
%
In this context, it is possible to avoid the complications of the complexity of
a non-perturbative calculation, resorting on a determination of the \pdfs
from experimental data.
In this way, a set of data used to determine the \pdfs can constrain the
predictions on other events, including different processes.
%
While the procedure is completely analogue to the determination of other \sm
model parameters, in the case of \pdfs there are two critical differences:
\begin{enumerate}
	\item the object determined is not an actual parameter of the theory, but
	      we need it because of the complexity of a first principles
	      determination - thus theory first principles theory calculation, able
	      to unfold the non-perturbative dynamics, like lattice, can contribute
	      to the \pdfs determination;
	\item the unknown parameter is not scalar, but a full function (actually a
	      finite set of them) over the $(0,1)$ domain, resulting in an infinite
	      amount of degrees of freedom to be determined.
\end{enumerate}
The second point will be further discussed in the context of \cref{ch:gp}.

\pdfs are not the only hadronic object arising from factorization, since also
final products observed in semi-inclusive measurements can be described by
close analogue, called \acrfull{ff}.
%
More complex object can describe the transverse (non-collinear) dynamics of
partons, like \acrfull{gpd} and \acrfull{tmd}.
Including more degrees of freedom, and requiring the measurement of more
differential observables, the state of this objects is still very raw, compared
to collinear \pdfs.
%
The research is actively ongoing in this field, but they will not be further
described in this thesis.

Moreover, other non-perturbative processes are involved in the predictions for
observables resulting from high-energy collisions.
In particular, the colored scattering products have to be grouped together in
hadrons, since no colored particle is eventually detected, because of \qcd
\textit{confinement}, another key property of the theory.
%
This mechanism is known as \textit{hadronization}, and it has a close interplay
with \acrfull{ps} and \qcd \textit{jets} observables.
%
Also this material is not treated in this thesis, but can be found in the
documentation of modern \acrlong{ps}s \cite{Bierlich:2022pfr,Bellm:2015jjp}.

\pdfs extraction is therefore essential for \hep research, especially for
hadronic machines like \lhc.
The methodology has been greatly improved in the past few years, and many more
processes are now contributing to their determination, together with
increasingly accurate theoretical predictions.
%
In the rest of this chapter, some key ingredients related to the \pdf
theoretical environment will be described.
\qcd Lagrangian and other fundamental properties are now textbook material, so
they will not described here, and instead the interested reader should refer to
well-known resources
\cite{Peskin:1995ev,Ellis:1996mzs,Campbell:2017hsr,Collins:2011zzd}.
