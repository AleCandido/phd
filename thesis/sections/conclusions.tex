% !TeX root = ../main.tex

%*******************************************************
% Introduzione
%*******************************************************
\cleardoublepage
\pdfbookmark{Conclusions}{conclusions}

\chapter*{Conclusions}
\markboth{\spacedlowsmallcaps{Conclusions}}%
	{\spacedlowsmallcaps{Conclusions}}

\historiated{T}{he} material presented cover a range of topics, all related to
\acrlong{pdf}, but about different aspects.
%
My PhD itself has been mainly focused on the rework of the theory predictions
architecture, in collaboration with other \nnpdf members, the number of which
increased during the years, together with projects' ambitions: the initial goal
of replacing the \dis and \dglap evolution modules was pursued by me and Felix
Hekhorn, but it grew to include a full rework of the theory pipeline, together
with other collaborators.

The modular structure of the architecture we created (the backbone of which we
inherited from \apfelcomb and previous \nnpdf projects) has been designed to
achieve three goals: extensibility, maintainability, and external access.
Indeed, while designed and developed by \nnpdf members, we hope that the other
groups interested in \pdf fitting, and related topics, can make use of the
tools we developed.
%
The development itself happened completely in public, and still continues this
way.
Until now, it has not been a big deal: projects were formally public, but not
being production-ready little interactions took place in practice.
However, we considered important to manifest from the beginning the will to
write tools at the disposal of the entire community, accepting feedback,
feature requests, and any kind of contributions (within the scope of the
projects).
%
This is motivated by the non-negligible expense in the amount of manpower
required, and the intention to lower the barrier for further studies, giving
free and easy access to common machinery.

We already benefited ourselves from the design choices made: the various
projects are the base on which new features are being introduced, as described
for the study of \acrlong{mhou} impact on \pdfs in \cref{ch:mhou}, and
additional progresses as well, even if they have not been accounted for in this
thesis, like the introduction of consistent \nnnlo theory predictions and other
elements already available in existing software (\qed corrections, polarized
elements or \ff-based predictions, small-$x$ resummed predictions), but all
provided by a single framework.
%
The individual modules are already being used in further studies, as it
happened for \eko in the quest for intrinsic charm in \cref{ch:ic}, or
\pineappl for \acrlong{dy} $A_{\text{fb}}$ exercise of \cref{ch:afb}, and also
\yadism, in the determination of low-energy neutrino structure functions (still
in preparation, and not described here).
As we are doing it, we hope that external users of our modules could the same
as well.
%
Also \pdf-wise, the new methodology proposal in \cref{ch:gp} is an example of
how having lowered the barrier through task decoupling allows for more
iterations on new ideas, and the consequential innovation.

Any new perturbative order is a completely new challenge, but this translates
in more complex calculations, heavier math in the results, and more
sophisticated approximations required for them (to keep them fast enough to
evaluate), while the rest of the infrastructure is ready to expand for new
contributions.
In particular, while \nnnlo inclusion has already been mentioned as a
work-in-progress effort, one of \pineappl initial motivations has been to
account for \ew corrections, and this will also be possible with current tools.
%
At the same time, also new data and new processes are becoming available, as
well as alternative data sources, like lattice computations, but the advantage
of the current approach is that we do not need any major modification of the
framework for them, but they will plugged as additional modules.

A specific effort in building the new architecture is being dedicated to
reproducibility, not only in making \textit{possible} to reobtain the same
result (thus reconstructing and scrutinizing all the choices made), but also
making it \textit{simple}, as much as possible, to encourage other people to do
it.
%
Running a \pdf fit, i.e.\ a fit accounting for higher orders \qcd effects on a
global data set, has always been a complex process, but it should become easier
over the years, despite more ingredients and further theory being added to the
present status, to encourage people to focus on the new developments, instead
of the complication of gathering all the pieces from scratch.

It is relevant to note that this last goal is not uniquely pursued by our
framework, since it exists at least the remarkable example of \xfitter
\url{https://www.xfitter.org/}, with a similar target, but a different
perspective, as it privileged an integrated (i.e.\ monolithic) architecture,
distributing theory predictions tools and the fitting machinery altogether.
Instead, we want to allow the user (possibly including \xfitter) to compute its
own theory predictions, and use them with their favorite fitting methodology.

All these tools are empowering \pdf fitters, in order to let them directing
efforts to the final purpose of more accurate and precise determinations,
hopefully leading to a fruitful collaboration to reduce the related theoretical
uncertainty still very relevant in many hadronic observables.
