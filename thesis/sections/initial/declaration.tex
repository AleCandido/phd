% !TeX root = ../../main.tex

%*******************************************************
% Declaration
%*******************************************************
\pdfbookmark{Declaration}{Declaration}

\chapter*{Declaration}
\markboth{\spacedlowsmallcaps{Declaration}}%
	{\spacedlowsmallcaps{Declaration}}

\noindent
This thesis is a report of the research activity conducted during my PhD.
%
Since part of this has already been published in papers, proceedings, or even
the documentation of the software developed, some of the material is already
appeared in those works.

Following, the description of the sources for each chapter.
\begin{description}[font=\normalfont\sffamily\scshape,leftmargin=2cm,style=nextline]
  \item[\cref{ch:dis}] content is adapted from the documentation of the \yadism
    package, \cite{candido_alessandro_2022_6285149}, available online at
    \url{https://yadism.readthedocs.io/}, and \cref{sec:dis/yadism} in
    particular has been initially written for a yet unpublished work on
    low-energy neutrino structure functions
  \item[\cref{ch:eko}]  mirrors the \eko paper, \cite{Candido:2022tld}, which
    in turn contains material from the \eko documentation,
    \url{https://eko.readthedocs.io/}, but some material in the docs has also
    been (and will be) backported from the paper itself
  \item[\cref{ch:pine}] is based on a proceeding appeared slightly before the
    thesis itself, \cite{Barontini:2022jci}, that is an early presentation of a
    work that will be discussed in a dedicated publication
  \item[\cref{ch:mhou}] has no public source, because the work is based on the
    toolchain exposed in the previous chapters, and the study itself has not
    yet reached its final stage; still, part of the material contained was
    originally authored as an internal note, for the \nnpdf collaboration's
    members, and adapted here for a (possibly) more generic audience
  \item[\cref{ch:ic}] reviews the content of a collaboration's result,
    \cite{Ball:2022qks}, based on \nnpdfr{4.0} release and \eko's features
  \item[\cref{ch:afb}] is based on an \nnpdf work, \cite{Ball:2022qtp}, based
    on \nnpdfr{4.0}, prompted by interaction with experimental users
  \item[\cref{ch:pos}] collects the material appeared in \cite{Candido:2020yat}
  \item[\cref{ch:gp}] presents a new work-in-progress methodology candidate,
    thus has no public reference at the moment, even though a likewise
    work-in-progress report exists online, \cite{petrillo2022}, but it is
    rather orthogonal to the content of the chapter, since focused on the
    technical implementation, while \cref{ch:gp} introduces just the gist of
    the idea, and sets the context
\end{description}
