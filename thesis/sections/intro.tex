% !TEX encoding = UTF-8
% !TEX TS-program = pdflatex
% !TEX root = ../Tesi.tex
% !TEX spellcheck = it-IT

%*******************************************************
% Introduzione
%*******************************************************
\cleardoublepage
\pdfbookmark{Introduction}{introduction}

\chapter*{Introduction}
\markboth{\spacedlowsmallcaps{Introduction}}%
	{\spacedlowsmallcaps{Introduction}}


\lettrine[lines=3]{\color{BrickRed}S}{ince} Einstein-Hilbert action is
non-renormalizable there is no natural candidate for a quantum theory of the
gravitation. Thus there are more different attempts to define a consistent
theory for Quantum Gravity, one of these is Asymptotic Safety
(\cite{Weinberg1979}).
\newline

Asymptotic Safety is still a Quantum Field Theory approach, the difference  is
that another notion of renormalizability is taken into account, that
generalizes the usual one; it consists in searching for a generic
Renormalization Group (RG) ultraviolet fixed points rather than requesting the
renormalizability perturbing around the Gaussian one.

Causal Dynamical Triangulation (CDT, \cite{Ambjorn2012}) is a non-perturbative
regulator for gravity used for defining the RG flow, in a way similar to that
of lattice in flat space QFT, in which the ultraviolet divergences are cutoffed
limiting the configurations (that in general is the quotient set of
differential manifolds with diffeomorphisms) to the piece-wise flat ones, where
the geometry is defined by an abstract triangulation and the length of the
simplex's edge, that acts as a lattice spacing.

Moreover a suitable version of discrete action must be defined to be evaluated
on a given triangulation, with the requirement to reproduce the
Einstein-Hilbert one in the continuum limit. This is called Regge action
(\cite{Regge1961}) and has been defined associating the curvature term to the
\textit{missing angles} (while the volume term is trivially the triangulation's
volume).
\newline

So CDT regulator allows the use of numerical simulations, because the new
configuration space has a discrete structure that can be represented in a
finite memory space.

This has been done employing Monte Carlo methods to create a sample distributed
according the probability defined by the Euclidean action, to obtain a discrete
version of the path integral. Some successful results have been found for a
particular phase, e.g. it has been shown that the shape of the simulated
quantum universe is consistent with a de Sitter one.
\newline

The main goal of the thesis is to extend the dynamical variables from the bare
geometry, i.e. a pure gravity theory, to the inclusion of gauge fields living
on the triangulation.
\newline

The first step consists in identifying an appropriate representation for the
new degrees of freedom in the framework of the Dynamical Triangulations. We
did this localizing gauge fields on the links of the graph dual to the
triangulation, and the reasons of this choice will be discussed in the thesis.

Then other two fundamental ingredients needed are: defining a suitable (and
efficient) Monte Carlo algorithm and an appropriate action that considers the
new degrees of freedom introduced.
\newline

The Monte Carlo algorithm for sample extraction is a local Markov Chain, in
the sense that transition probability are non-zero only between those
configurations that differ only in a small region.
The small regions considered are called \textit{cell}, and are the smallest
available to jump from a given triangulation in another legal one. These jumps,
i.e. the set of couples of "adjacent" triangulations on which the Markov Chain
has non-zero values, are called \textit{moves}, and are already defined in
absence of gauge fields.

The presence of the gauge fields require mainly two upgrades of the Markov
Chain: the definition of gauge structure update during each triangulation move
and a dedicated gauge move (performed on a fixed triangulation) that updates a
single gauge field, for which we choose to employ the heat-bath algorithm.
The second upgrade is needed in order to ensure the ergodicity of the upgraded
Markov Chain.
\newline

We define the second ingredient, the action for gauge-including triangulations,
merging the most similar objects that are already used: the Regge action for
triangulations and the lattice one for gauge theories.

While the Regge action is already appropriate for this kind of objects the
lattice action needs to be generalized for being applied on triangulations,
because its definition is for regular flat lattices.
This generalization has been chosen in order to still have a correct naïve
continuum limit, and to be proportional to the original action on a flat
triangulation; this is done multiplying the term preexisting term for the
inverse length of the plaquette, that oppositely to what happens for hypercubic
lattices is not fixed for generic triangulations, and according the Regge
action is the measure of the local curvature.
\newline

We used the defined algorithm to test the behavior of gauge fields in a
dynamical geometry, and their back-reaction on the geometry itself.
So both gauge and triangulation observables are analyzed to characterize the
physics of the system, in particular, the causal structure of CDT gives us a
straightforward definition of correlation lengths, and taking one length from
the geometry (from slice volumes' correlation) and one from the gauge sector
(torelons for example) it is possible to define a common continuum limit, where
the two lengths scale with a fixed ratio while taking triangulations of
increasing volume.

\vspace{20pt}

The thesis is structured in five chapters: the first three are an introduction
to theory and the methods we used, while the following two contain our
developments.

In particular:
\begin{description}
	\item[{\hyperref[cap:assty]{The first chapter}}] is a review of the
		Asymptotic Safety physical principle, and some related concepts are
		introduced;
	\item[{\hyperref[cap:cdt]{The second chapter}}] is a review of the Causal
		Dynamical Triangulation framework, explaining how and why it can be
		used as a theoretical and numerical instrument for studying quantum
		gravity;
	\item[{\hyperref[cap:nummet]{The third chapter}}] is an introduction to
		some numerical methods employed in our simulations;
	\item[{\hyperref[cap:gcdt]{The fourth chapter}}] is about the introduction
		of gauge fields dynamics in CDT and it describes both the theoretical
		principles and the new Monte Carlo algorithm;
	\item[{\hyperref[cap:res]{The fifth chapter}}] collects the results of the
		simulations performed with the new algorithm, discussing the effects of
		gauge on triangulations dynamics and vice versa;
	\item[{\hyperref[cap:misc]{The appendix A}}] analyzes in details some
		concepts introduced in \cref{cap:assty,cap:cdt};
	\item[{\hyperref[cap:gaugeapp]{The appendix B}}] contains detailed
		description of topics related to gauge fields, introduced in
		\cref{cap:gcdt} and used in simulations.
\end{description}
