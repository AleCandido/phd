% !TeX program = pdflatex
% !TeX root = ../main.tex

%*******************************************************
% Introduzione
%*******************************************************
\cleardoublepage
\pdfbookmark{Introduction}{introduction}

\chapter*{Introduction}
\markboth{\spacedlowsmallcaps{Introduction}}%
	{\spacedlowsmallcaps{Introduction}}

\historiated{D}{uring} my doctorate, I have been involved in projects regarding
different topics, all related to the common theme of \acrfull{pdf}, and their
ecosystem, half-way between theoretical and experimental \acrfull{hep}, but
whose challenges extend to statistics, and deeply into the world of software.

As it is possible to infer from the thesis structure, the main subject
consists in constructing suitable theory predictions, in order to allow some
new studies in the \acrlong{nnpdf}, and creating a consistent and improved
framework, by designing extensible tools, that will simplify the inclusion of
new physical observables and improved theory calculations.
This will be discussed extensively in the \textit{first part} of the thesis.

While this has been the main focus for a long time, when me and my
collaborators started approaching the completion of the first prototype
framework, and the integration in the main workflow of \acrshort{nnpdf}, some
new options for dedicated studies became immediately available.
These applications are collected and discussed in the \textit{second part} of
the thesis.
Most of them are connected not only to the availability of the tools we
created, but also to the many achievements and past works of \acrshort{nnpdf},
that created a unique foundation for many studies, given the flexibility of its
novel methodology and the unparalleled precision reached with its extensive
datasets, including experiments spanning multiple decades, and several
different processes, from lepton-hadron and hadron-hadron colliders and
fixed-target experiments.
