% !TeX program = pdflatex
% !TeX root = ../main.tex

%*******************************************************
% Introduzione
%*******************************************************
\cleardoublepage
\pdfbookmark{Introduction}{introduction}

\chapter*{Introduction}
\markboth{\spacedlowsmallcaps{Introduction}}%
	{\spacedlowsmallcaps{Introduction}}

\historiated{T}{he} common thread of this thesis are \acrfull{pdf} and their
ecosystem, half-way between theoretical and experimental \acrfull{hep}: being
strongly data-driven, they greatly depend on experiments precision and results
availability.
But theory is also crucial for the extraction, since \acrshortpl{pdf} are
determined to best fit data with theoretical predictions.

As it is possible to infer from the thesis structure, the main subject consists
in constructing suitable predictions, in order to allow some new studies
in the \acrlong{nnpdf}, and creating a consistent and improved framework, by
designing extensible tools, that will simplify the inclusion of new physical
observables and improved theory calculations.
This will be discussed extensively in the \hyperref[part:th]{\textit{first
part}} of the thesis, and it will include a package dedicated to the solution
of \dglap equations \cref{ch:eko}, another providing \acrfull{dis} predictions
and grids \cref{ch:dis}, and the discussion of the full framework cited
\cref{ch:pine}, where the two packages are integrated in.

While this has been the main focus for a long time, when me and my
collaborators started approaching the completion of the first prototype
framework, and the integration in the main \acrshort{nnpdf} workflow, some new
options for dedicated studies became immediately available.
These applications are collected and discussed in the
\hyperref[part:app]{\textit{second part}} of the thesis.
Most of them are connected not only to the availability of the tools we
created, but also to the many achievements and past works of \acrshort{nnpdf},
that created a unique foundation for many studies, given the flexibility of its
novel methodology and the unparalleled precision reached with its extensive
datasets, including experiments spanning multiple decades, and several
different processes, from lepton-hadron and hadron-hadron colliders and
fixed-target experiments.

Moreover, extracting \pdfs is deeply connected to the statistical methods
applied.
This is more or less true for any analysis based on quantitative experimental
data, but it is particularly relevant for the case of \pdfs, because of the
functional nature of the objected extracted, that exacerbates the dependency of
the result on the methodology.
From this perspective, the \nnpdf methodology is already a novelty, since it
required a suitable extension of the usual statistical treatment, based on a
given parametrization, in order to access more complex model developed by the
Machine Learning community, where direct control of parameter space is
difficult, and not very useful.
The whole procedure has been recently innovated, even if I have not taken part
directly to this process, that finally led to the release of \nnpdfr{4.0}
\cite{NNPDF:2021njg}, I investigated some limitations of the current approach,
especially considering the perspective of the full distribution, that in the
\nnpdf methodology is only arising at the end of the whole fit, but not used in
the individual optimization steps.
This led to the proposal of a new candidate methodology, described in
\cref{ch:gp}, that would replace the current usage of a \acrfull{nn} with
different techniques.
Nevertheless, we currently conjecture the final result to be mostly compatible,
but at the time of writing the proposal is still at an early stage, so no full
check has yet been performed.

This topic is collected in the \hyperref[part:meth]{\textit{third part}},
together with an investigation about the positivity of certain \pdfs,
described in \cref{ch:pos}.
The reason why the two things are bundled together is that they both affect the
final extraction methodology, even if in two very different ways.
In \nnpdf itself, three main lines of development have always been identified:
data implementation, theory computation and extension, and methodological
improvements.
If the first part is  mainly devoted to the theory, this one is instead
connected to the methodology.

A final remark is required: software development has been a big share of the
main effort.
While in the first part this is manifest, it is actually underlying any work
described, even though not  always to the same extent.
The whole \nnpdf architecture, and collaboration's main results, are deeply
connected to the development of increasingly more reliable tools.
Eventually, the main code has been published \cite{NNPDF:2021uiq}, in order to
support full transparency, and to make it available for more studies.
Potentially, even by authors external to the collaboration.
Following this philosophy, all the projects I took part in have been developed
open immediately, and they are available in the \ghurl{NNPDF} GitHub
organization (a few minor ones still in the \ghurl{N3PDF} organization), with
special care for usability and maintainability, to the best of our abilities.
