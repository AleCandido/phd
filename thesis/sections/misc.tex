% !TEX encoding = UTF-8
% !TEX TS-program = pdflatex
% !TEX root = ../main.tex
% !TEX spellcheck = en-US

%************************************************
\chapter{Miscellanea}
\label{cap:misc}
%************************************************

\minitoc
\adjustmtc

%\vspace*{10pt}
%\ttmp{Un altro nome possibile è qualcosa tipo \textbf{Letteratura}, per specificare che è riferito ai conti della prima parte.}

\section{Asymptotic Safety}

In this section we describe in more details some ideas cited in \cref{cap:AsSty}, taken from \cite{Weinberg1979}.

\subsection{Unphysical poles in propagators}
\label{ssec:upoles}

The advantage of the new propagator is to be more regular in the ultraviolet regime, vanishing with an higher power of $ k $ (i.e. $ k^{-4} $), and for this reason the theory passes the usual power-counting test for renormalizability,
But now new unphysical poles are present in the propagator at $ k^2 = -1/{\alpha} $, with the wrong sign for unitarity consistency.
